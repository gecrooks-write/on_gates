
% !TEX encoding = UTF-8 Unicode 
% !TEX root = on_gates.tex

\clearpage

\section{Standard 3-qubit gates}

Regrettable there doesn't appear to be an easy way to characterize and visualizes the space of 3-qubit gates in the same way there is for 1-qubit (Bloch ball) and 2-qubit gates (Weyl chamber). Which is perhaps not surprising since a general 3-qubit gate has $(2^3)^2 = 64$ parameters. 

Fortunately there are only a few specific 3-qubit gates that show up in practice, most of which are directly related to the Toffoli (or controlled-controlled-not) gate.

\paragraph{Toffoli gate (controlled-controlled-not, CCNot)}\cite{Toffoli1980a, Feynman1985a, Barenco1995b}
\index{CCNot gate}
\index{controlled-controlled-not gate|see{CCNot gate}}
\index{Toffoli gate|see{CCNot gate}}
A 3-qubit gate with two control and one target qubits.  Originally studied in the context of reversible classical logic, where 3-bit gates are necessary for universal computation~\cite{Toffoli1980a,Aaronson2015}. %1504.05155
 The target bit flips only if both control bits are one.  We often encounter this gate when converting classical logic circuits to quantum circuits.
$$
\text{CCNot} =
\begin{bsmallmatrix}
 1 & 0 & 0 & 0 & 0 & 0 & 0 & 0 \\
 0 & 1 & 0 & 0 & 0 & 0 & 0 & 0 \\
 0 & 0 & 1 & 0 & 0 & 0 & 0 & 0 \\
 0 & 0 & 0 & 1 & 0 & 0 & 0 & 0 \\
 0 & 0 & 0 & 0 & 1 & 0 & 0 & 0 \\
 0 & 0 & 0 & 0 & 0 & 1 & 0 & 0 \\
 0 & 0 & 0 & 0 & 0 & 0 & 0 & 1 \\
 0 & 0 & 0 & 0 & 0 & 0 & 1 & 0 \\
\end{bsmallmatrix}
\quad = \ \adjustbox{scale=0.8}{\begin{quantikz}[thin lines, column sep=0.75em,row sep={2.5em,between origins}]
& \ctrl{1} & \qw \\
& \ctrl{1} & \qw \\
& \targ{} & \qw
\end{quantikz}
}
$$
\newcommand{\sm}{{\text{-}}} % Small minus
\[
\ham{CCNot} & = -\tfrac{\pi}{8}(I_0-Z_0)(I_1-Z_1)(I_2-X_2) \\
& =
\tfrac{\pi}{2} \begin{bsmallmatrix*}[r]
 0 & 0 & 0 & 0 & 0 & 0 & 0 & 0 \\
 0 & 0 & 0 & 0 & 0 & 0 & 0 & 0 \\
 0 & 0 & 0 & 0 & 0 & 0 & 0 & 0 \\
 0 & 0 & 0 & 0 & 0 & 0 & 0 & 0 \\
 0 & 0 & 0 & 0 & 0 & 0 & 0 & 0 \\
 0 & 0 & 0 & 0 & 0 & 0 & 0 & 0 \\
 0 & 0 & 0 & 0 & 0 & 0 & \sm 1 & 1 \\
 0 & 0 & 0 & 0 & 0 & 0 & 1 & \sm 1 \\
\end{bsmallmatrix*}
\notag
\]



%The CCNot gate is typically represented by the following circuit:
%$$
%\adjustbox{scale=0.8}{\begin{quantikz}[thin lines, column sep=0.75em,row sep={2.5em,between origins}]
& \ctrl{1} & \qw \\
& \ctrl{1} & \qw \\
& \targ{} & \qw
\end{quantikz}
}
%$$

The CCNot can be decomposed into a circuit of at least 6 CNots~\cite{Nielsen2000a}.
$$\adjustbox{scale=0.8}{\begin{quantikz}[thin lines, column sep=0.75em,row sep={2.5em,between origins}]
& \ctrl{1} & \qw \\
& \ctrl{1} & \qw \\
& \targ{} & \qw
\end{quantikz}
} \simeq
\adjustbox{scale=0.8}{\begin{quantikz}[thin lines, column sep=0.75em,row sep={2.5em,between origins}]
& \qw & \qw & \qw & \ctrl{2} & \qw & \qw & \qw & \ctrl{2} & \qw & \ctrl{1} & \gate{T} & \ctrl{1} & \qw \\
& \qw & \ctrl{1} & \qw & \qw & \qw & \ctrl{1} & \qw & \qw & \gate{T} & \targ{} & \gate{T^\dagger} & \targ{} & \qw \\
& \gate{H} & \targ{} & \gate{T^\dagger} & \targ{} & \gate{T} & \targ{} & \gate{T^\dagger} & \targ{} & \qw & \qw & \gate{T} & \gate{H} & \qw
\end{quantikz}
}
$$
The above circuit assumes that we can apply CNot gates between any of the 3 qubits. If we are instead restricted to CNots between adjacent qubits, then we can decompose into 8 \Gate{CNot} gates, which is fewer than if we added explicit Swap operations. 
 $$\adjustbox{scale=0.8}{\begin{quantikz}[thin lines, column sep=0.75em,row sep={2.5em,between origins}]
& \ctrl{1} & \qw \\
& \ctrl{1} & \qw \\
& \targ{} & \qw
\end{quantikz}
} \simeq
\adjustbox{scale=0.8}{\begin{quantikz}[thin lines, column sep=0.75em,row sep={2.5em,between origins}]
& \gate{T} & \ctrl{1} & \qw & \qw & \ctrl{1} & \qw & \ctrl{1} & \qw & \ctrl{1} & \qw & \qw & \qw \\
& \gate{T} & \targ{} & \ctrl{1} & \gate{T^\dagger} & \targ{} & \ctrl{1} & \targ{} & \ctrl{1} & \targ{} & \ctrl{1} & \qw & \qw \\
& \gate{H} & \gate{T} & \targ{} & \gate{T} & \qw & \targ{} & \gate{T^\dagger} & \targ{} & \gate{T^\dagger} & \targ{} & \gate{H} & \qw
\end{quantikz}
}
$$
This depth 9 decomposition requires 7 CNots~\cite{Amy2013a}. Since more gates can be applied at the same time, the gate depth is less despite more 2-qubit gates.
% Minimum T-depth of 3
$$\adjustbox{scale=0.8}{\begin{quantikz}[thin lines, column sep=0.75em,row sep={2.5em,between origins}]
& \ctrl{1} & \qw \\
& \ctrl{1} & \qw \\
& \targ{} & \qw
\end{quantikz}
} \simeq
\adjustbox{scale=0.8}{\begin{quantikz}[thin lines, column sep=0.75em,row sep={2.5em,between origins}]
& \gate{T} & \targ{} & \qw & \qw & \ctrl{2} & \ctrl{1} & \qw & \gate{T^\dagger} & \ctrl{2} & \targ{} & \qw \\
& \gate{T} & \ctrl{-1} & \targ{} & \gate{T^\dagger} & \qw & \targ{} & \gate{T^\dagger} & \targ{} & \qw & \ctrl{-1} & \qw \\
& \gate{H} & \gate{T} & \ctrl{-1} & \qw & \targ{} & \qw & \gate{T} & \ctrl{-1} & \targ{} & \gate{H} & \qw
\end{quantikz}
}
$$
Another decomposition requires 3 \Gate{CV} and 2 \Gate{CNot} gates~\cite{???}.
$$\adjustbox{scale=0.8}{\begin{quantikz}[thin lines, column sep=0.75em,row sep={2.5em,between origins}]
& \ctrl{1} & \qw \\
& \ctrl{1} & \qw \\
& \targ{} & \qw
\end{quantikz}
} \simeq
\adjustbox{scale=0.8}{\begin{quantikz}[thin lines, column sep=0.75em,row sep={2.5em,between origins}]
& \qw & \ctrl{1} & \qw & \ctrl{1} & \ctrl{2} & \qw \\
& \ctrl{1} & \targ{} & \ctrl{1} & \targ{} & \qw & \qw \\
& \gate{V} & \qw & \gate{V^\dagger} & \qw & \gate{V} & \qw
\end{quantikz}
}
$$



\paragraph{Fredkin gate (controlled-swap, CSwap)}\cite{Fredkin1982a,???}
\index{controlled-swap gate}
\index{CSwap gate|see{controlled-swap gate}}
\index{Fredkin gate|see{controlled-swap gate}}
$$
\text{CSwap} = 
\begin{bsmallmatrix}
 1 & 0 & 0 & 0 & 0 & 0 & 0 & 0 \\
 0 & 1 & 0 & 0 & 0 & 0 & 0 & 0 \\
 0 & 0 & 1 & 0 & 0 & 0 & 0 & 0 \\
 0 & 0 & 0 & 1 & 0 & 0 & 0 & 0 \\
 0 & 0 & 0 & 0 & 1 & 0 & 0 & 0 \\
 0 & 0 & 0 & 0 & 0 & 0 & 1 & 0 \\
 0 & 0 & 0 & 0 & 0 & 1 & 0 & 0 \\
 0 & 0 & 0 & 0 & 0 & 0 & 0 & 1 \\
\end{bsmallmatrix}
\quad = \
\adjustbox{scale=0.8}{\begin{quantikz}[thin lines, column sep=0.75em,row sep={2.5em,between origins}]
& \ctrl{1} & \qw \\
& \swap{1} & \qw \\
& \targX{} & \qw
\end{quantikz}
}
$$
A controlled swap gate. Another logic gate for reversible classical computing.
\[
\ham{CSwap} & = -\tfrac{\pi}{8}(I_0-Z_0)(X_1 X_2 + Y_1 Y_2 + Z_1 Z_2 - I_1 I_2)\\
& =
\tfrac{\pi}{2} \begin{bsmallmatrix*}[r]
 0 & 0 & 0 & 0 & 0 & 0 & 0 & 0 \\
 0 & 0 & 0 & 0 & 0 & 0 & 0 & 0 \\
 0 & 0 & 0 & 0 & 0 & 0 & 0 & 0 \\
 0 & 0 & 0 & 0 & 0 & 0 & 0 & 0 \\
 0 & 0 & 0 & 0 & 0 & 0 & 0 & 0 \\
 0 & 0 & 0 & 0 & 0 & \text{-}1 & 1 & 0 \\
 0 & 0 & 0 & 0 & 0 & 1 & \text{-}1 & 0 \\
 0 & 0 & 0 & 0 & 0 & 0 & 0 & 0 \\
\end{bsmallmatrix*}
\notag
\]
A CSwap can be built from 2 CNot gates and 1 CCNot (or 8 CNots in total).
$$
\adjustbox{scale=0.8}{\begin{quantikz}[thin lines, column sep=0.75em,row sep={2.5em,between origins}]
& \ctrl{1} & \qw \\
& \swap{1} & \qw \\
& \targX{} & \qw
\end{quantikz}
} \simeq \adjustbox{scale=0.8}{\begin{quantikz}[thin lines, column sep=0.75em,row sep={2.5em,between origins}]
& \qw & \ctrl{1} & \qw & \qw \\
& \targ{} & \ctrl{1} & \targ{} & \qw \\
& \ctrl{-1} & \targ{} & \ctrl{-1} & \qw
\end{quantikz}
} 
$$
An adjacency respecting decomposition of the CSwap can be formed with 10 CNots if the target is the first qubit~\cite{???},
$$
\adjustbox{scale=0.8}{\begin{quantikz}[thin lines, column sep=0.75em,row sep={2.5em,between origins}]
& \ctrl{1} & \qw \\
& \swap{1} & \qw \\
& \targX{} & \qw
\end{quantikz}
} \simeq \adjustbox{scale=0.8}{\begin{quantikz}[thin lines, column sep=0.75em,row sep={2.5em,between origins}]
& \gate{T} & \qw & \ctrl{1} & \qw & \qw & \qw & \ctrl{1} & \qw & \ctrl{1} & \qw & \qw & \ctrl{1} & \qw & \qw & \qw & \qw \\
& \targ{} & \gate{T} & \targ{} & \qw & \ctrl{1} & \gate{T^\dagger} & \targ{} & \ctrl{1} & \targ{} & \ctrl{1} & \gate{V} & \targ{} & \ctrl{1} & \qw & \gate{V} & \qw \\
& \ctrl{-1} & \gate{H} & \gate{S^\dagger} & \gate{T^\dagger} & \targ{} & \gate{T} & \qw & \targ{} & \gate{T^\dagger} & \targ{} & \qw & \gate{T^\dagger} & \targ{} & \gate{S} & \gate{V^\dagger} & \qw
\end{quantikz}
} 
$$
or 12 CNots if the target is between the two swapped qubits~\cite{???}.
$$
\adjustbox{scale=0.8}{\begin{quantikz}[thin lines, column sep=0.75em,row sep={2.5em,between origins}]
& \ctrl{1} & \qw \\
& \swap{1} & \qw \\
& \targX{} & \qw
\end{quantikz}
} \simeq \adjustbox{scale=0.8}{\begin{quantikz}[thin lines, column sep=0.75em,row sep={2.5em,between origins}]
& \targ{} & \ctrl{1} & \targ{} & \qw & \qw & \qw & \ctrl{2} & \gate{T^\dagger} & \targ{} & \ctrl{2} & \gate{T} & \targ{} & \ctrl{2} & \gate{V} & \ctrl{1} & \ctrl{2} & \targ{} & \ctrl{2} & \qw & \qw & \qw \\
& \ctrl{-1} & \targ{} & \qw & \gate{T} & \qw & \qw & \qw & \qw & \ctrl{-1} & \qw & \qw & \ctrl{-1} & \qw & \qw & \targ{} & \qw & \ctrl{-1} & \qw & \qw & \gate{V^\dagger} & \qw \\
& \qw & \qw & \ctrl{-2} & \qw & \gate{H} & \gate{T} & \targ{} & \gate{T} & \qw & \targ{} & \qw & \gate{T^\dagger} & \targ{} & \qw & \gate{T^\dagger} & \targ{} & \qw & \targ{} & \gate{H} & \gate{S^\dagger} & \qw
\end{quantikz}
} 
$$

\paragraph{CCZ gate (controlled-controlled-Z)}~\cite{???,???}
\index{CCZ gate}
\index{controlled-controlled-Z|see{CCZ gate}}
$$
CCZ=
\begin{bsmallmatrix*}[r]
 1 & 0 & 0 & 0 & 0 & 0 & 0 & 0 \\
 0 & 1 & 0 & 0 & 0 & 0 & 0 & 0 \\
 0 & 0 & 1 & 0 & 0 & 0 & 0 & 0 \\
 0 & 0 & 0 & 1 & 0 & 0 & 0 & 0 \\
 0 & 0 & 0 & 0 & 1 & 0 & 0 & 0 \\
 0 & 0 & 0 & 0 & 0 & 1 & 0 & 0 \\
 0 & 0 & 0 & 0 & 0 & 0 & 1 & 0 \\
 0 & 0 & 0 & 0 & 0 & 0 & 0 & \sm 1 \\
\end{bsmallmatrix*}
\quad = \ \adjustbox{scale=0.8}{\begin{quantikz}[thin lines, column sep=0.75em,row sep={2.5em,between origins}]
& \ctrl{1} & \qw \\
& \ctrl{1} & \qw \\
& \gate{Z} & \qw
\end{quantikz}
}
$$
\[
\ham{CCZ} & = -\tfrac{\pi}{8}(I_0-Z_0)(I_1-Z_1)(I_2-Z_2) \\
& =
\tfrac{\pi}{2} \begin{bsmallmatrix*}[r]
 0 & 0 & 0 & 0 & 0 & 0 & 0 & 0 \\
 0 & 0 & 0 & 0 & 0 & 0 & 0 & 0 \\
 0 & 0 & 0 & 0 & 0 & 0 & 0 & 0 \\
 0 & 0 & 0 & 0 & 0 & 0 & 0 & 0 \\
 0 & 0 & 0 & 0 & 0 & 0 & 0 & 0 \\
 0 & 0 & 0 & 0 & 0 & 0 & 0 & 0 \\
 0 & 0 & 0 & 0 & 0 & 0 & 0 & 0 \\
 0 & 0 & 0 & 0 & 0 & 0 & 0 &  \sm1 \\
\end{bsmallmatrix*}
\notag
\]
\todo{[TODO: Checkme hamiltonian]}
The CCNot gate can be converted to the CCZ gate by conjugating the target qubit with Hadamard gates (in the same way that we can convert a CNot to CZ)
$$
\adjustbox{scale=0.8}{\begin{quantikz}[thin lines, column sep=0.75em,row sep={2.5em,between origins}]
& \ctrl{1} & \qw \\
& \ctrl{1} & \qw \\
& \gate{Z} & \qw
\end{quantikz}
} \simeq \adjustbox{scale=0.8}{\begin{quantikz}[thin lines, column sep=0.75em,row sep={2.5em,between origins}]
& \qw & \ctrl{1} & \qw & \qw \\
& \qw & \ctrl{1} & \qw & \qw \\
& \gate{H} & \targ{} & \gate{H} & \qw
\end{quantikz}
} 
$$
\todo{[TODO: Fix circuit diagram of CCZ] [TODO: Can move target]}


\paragraph{Peres gate}~\cite{Peres1985a, Maslov2003a}\index{Peres gate} Another gate that is universal for classical reversible computing.
\[
\text{Peres} = 
\Left[ \begin{smallmatrix}
                1& 0& 0& 0& 0& 0& 0& 0 \\
                0& 1& 0& 0& 0& 0& 0& 0 \\
                0& 0& 1& 0& 0& 0& 0& 0 \\
                0& 0& 0& 1& 0& 0& 0& 0 \\
                0& 0& 0& 0& 0& 0& 0& 1 \\
                0& 0& 0& 0& 0& 0& 1& 0 \\
                0& 0& 0& 0& 0& 1& 0& 0 \\
                0& 0& 0& 0& 1& 0& 0& 0 
\end{smallmatrix} \Right]
\]
The Peres gate is equivalent to a Toffoli followed by a CNot gate, and decomposes into 5 CNots (compared to 6 for Toffoli gates).
$$
\adjustbox{scale=0.8}{\begin{quantikz}[thin lines, column sep=0.75em,row sep={2.5em,between origins}]
& \ctrl{1} & \ctrl{1} & \qw \\
& \ctrl{1} & \targ{} & \qw \\
& \targ{} & \qw & \qw
\end{quantikz}
}
$$
% Cite https://arxiv.org/pdf/quant-ph/0403053.pdf
% https://arxiv.org/pdf/0803.2316.pdf
\newcommand*{\circledbullet}{%
    \mathbin{%
        \ooalign{$\circledcirc$\cr\hidewidth$\bullet$\hidewidth}%
    }%
}
The Peres gate is a reversible half-adder. (Recall that a half-adder sums two bits, whereas a full-adder sums three bits) If we feed a zero bit into the third position, then the output of the second bit is the sum (mod 2) of the first two bits, and the third bit is the carry. 
$$
\adjustbox{scale=0.8}{\begin{quantikz}[thin lines, column sep=0.75em,row sep={2.5em,between origins}]
\lstick{A} & \ctrl{2} & \qw & \rstick{A}\\
\lstick{B} & \push{\circledbullet}  & \qw  &\rstick{\text{sum}}\\
\lstick{0}& \targ{}  & \qw & \rstick{\text{carry}}
\end{quantikz}
}
$$
The above diagram is seen occasionally, where the middle bit of the Peres gate is denoted by a fisheye.



\paragraph{Deutsch gate}~\cite{Deutsch1989a, Barenco1995a, Shi2018a}\index{Deutsch gate}
\index{Deutsch gate}
 Mostly of historical interest, since this was the first quantum gate to be shown to be computationally universal~\cite{Deutsch1989a}. 
\[
\label{Deutsch}
\Gate{Deutsch}(\theta) =
\begin{bsmallmatrix}
                1& 0& 0& 0& 0& 0& 0& 0 \\
                0& 1& 0& 0& 0& 0& 0& 0 \\
                0& 0& 1& 0& 0& 0& 0& 0 \\
                0& 0& 0& 1& 0& 0& 0& 0 \\
                0& 0& 0& 0& 1& 0& 0& 0 \\
                0& 0& 0& 0& 0& 1& 0& 0 \\
                0& 0& 0& 0& 0& 0& i \cos(\theta) & \sin(\theta) \\
                0& 0& 0& 0& 0& 0& \sin(\theta)& i \cos(\theta)
\end{bsmallmatrix}
\]
Examining the controlled unitary sub-matrix, the Deutsch gate can be thought of as a controlled-controlled-$i\Gate{R_x}^2(\theta)$ gate.
$$
\text{Deutsch}(\theta) = 
\input{circuits/Deutsch}
$$

Barenco~\cite{Barenco1995a} demonstrated a construction of the Deutsch gate from 2-qubit ``Barenco'' gates, demonstrating that a single type of 2-qubit gate is sufficient for universality.\index{Barenco gate}
$$
 \input{circuits/Deutsch}
 \simeq 
\adjustbox{scale=0.75}{\begin{quantikz}[thin lines, column sep=0.75em,row sep={2.5em,between origins}]
& \qw & \ctrl{2} & \ctrl{1} & \qw & \ctrl{1} & \qw \\
& \ctrl{1} & \qw & \gate{\text{Bar}(0, \tfrac{\pi}{2}, \tfrac{\pi}{2})} & \ctrl{1} & \gate{\text{Bar}(0, \tfrac{\pi}{2}, \tfrac{\pi}{2})} & \qw \\
& \gate{\text{Bar}(0, \tfrac{\pi}{4}, \tfrac{\theta}{2})} & \gate{\text{Bar}(0, \tfrac{\pi}{4}, \tfrac{\theta}{2})} & \qw & \gate{\text{Bar}(\pi, -\tfrac{\pi}{4}, \tfrac{\theta}{2})} & \qw & \qw
\end{quantikz}
} 
% \adjustbox{scale=0.8}{\begin{quantikz}[thin lines, column sep=0.75em,row sep={2.5em,between origins}]
& \qw & \ctrl{2} & \ctrl{1} & \qw & \ctrl{1} & \qw \\
& \ctrl{1} & \qw & \gate{\text{Bar}(0, \frac{\pi}{2}, \frac{\pi}{2})} & \ctrl{1} & \gate{\text{Bar}(0, \frac{\pi}{2}, \frac{\pi}{2})} & \qw \\
& \gate{\text{Bar}(0, \frac{\pi}{4}, \frac{\theta}{2})} & \gate{\text{Bar}(0, \frac{\pi}{4}, \frac{\theta}{2})} & \qw & \gate{\text{Bar}(\pi, - \frac{\pi}{4}, \frac{\theta}{2})} & \qw & \qw
\end{quantikz}
}
 $$


% Also cite: https://journals.aps.org/prapplied/abstract/10.1103/PhysRevApplied.9.051001
% for proposed physical implementation Shi2018a


% QFT: Quantum Fourier transform,=. Not to be confuesd wiht, in different conetxts Quantu Field Theory, or Quanutm Flutuation theorem~\cite{Theotherqft}. At least Q alwyas standas for Quantum.


\paragraph{CCiX gate}~\cite{Selinger2013a, Quipper???, Maslov2016a}
\index{CCiX gate}\index{doubly controlled iX gate|see{CCiX gate}} A doubly controlled iX gate.
\[
        \Gate{CCiX} &= \begin{bsmallmatrix}
                1 & 0 & 0 & 0 & 0 & 0 & 0 & 0 \\
                0 & 1 & 0 & 0 & 0 & 0 & 0 & 0 \\
                0 & 0 & 1 & 0 & 0 & 0 & 0 & 0 \\
                0 & 0 & 0 & 1 & 0 & 0 & 0 & 0 \\
                0 & 0 & 0 & 0 & 1 & 0 & 0 & 0 \\
                0 & 0 & 0 & 0 & 0 & 1 & 0 & 0 \\
                0 & 0 & 0 & 0 & 0 & 0 & 0 & i \\
                0 & 0 & 0 & 0 & 0 & 0 & i & 0
            \end{bsmallmatrix}
            \label{CCiX}
\\ \notag
 H_\Gate{CCiX} &=-\frac{\pi}{8} X_2 (1 - Z_1) (1 - Z_0)            
\]
Can be decomposed into 4 CNot gates,
$$
\adjustbox{scale=0.8}{\begin{quantikz}[thin lines, column sep=0.75em,row sep={2.5em,between origins}]
& \ctrl{1} & \qw \\
& \ctrl{1} & \qw \\
& \gate{iX} & \qw
\end{quantikz}
} \simeq
\adjustbox{scale=0.8}{\begin{quantikz}[thin lines, column sep=0.75em,row sep={2.5em,between origins}]
& \qw & \qw & \qw & \ctrl{2} & \qw & \qw & \qw & \ctrl{2} & \qw & \qw & \qw \\
& \qw & \ctrl{1} & \qw & \qw & \qw & \ctrl{1} & \qw & \qw & \qw & \qw & \qw \\
& \gate{H} & \targ{} & \gate{T} & \targ{} & \gate{T^\dagger} & \targ{} & \gate{T} & \targ{} & \gate{T^\dagger} & \gate{H} & \qw
\end{quantikz}
}
$$
or 8 CNots respecting adjacency~\cite{Selinger2013a, Green2013a}.
$$
\adjustbox{scale=0.8}{\begin{quantikz}[thin lines, column sep=0.75em,row sep={2.5em,between origins}]
& \ctrl{1} & \qw \\
& \ctrl{1} & \qw \\
& \gate{iX} & \qw
\end{quantikz}
} \simeq
\adjustbox{scale=0.8}{\begin{quantikz}[thin lines, column sep=0.75em,row sep={2.5em,between origins}]
& \qw & \qw & \targ{} & \gate{T^\dagger} & \qw & \targ{} & \gate{T} & \targ{} & \qw & \targ{} & \qw \\
& \qw & \targ{} & \ctrl{-1} & \gate{T} & \targ{} & \ctrl{-1} & \targ{} & \ctrl{-1} & \targ{} & \ctrl{-1} & \qw \\
& \gate{H} & \ctrl{-1} & \qw & \qw & \ctrl{-1} & \gate{T^\dagger} & \ctrl{-1} & \qw & \ctrl{-1} & \gate{H} & \qw
\end{quantikz}
}
$$

%    Refs:
%          http://arxiv.org/abs/1210.0974
%    """
%    # Kudos: Adapted from 1210.0974, via Quipper
%

\paragraph{CiSwap gate}\index{CiSwap gate} A controlled iSwap gate. 
\index{CiSwap gate|see{controlled iSwap gate}}\index{controlled iSwap gate}
\[
        \Gate{CiSwap} &= \begin{bsmallmatrix}
                1 & 0 & 0 & 0 & 0 & 0 & 0 & 0 \\
                0 & 1 & 0 & 0 & 0 & 0 & 0 & 0 \\
                0 & 0 & 1 & 0 & 0 & 0 & 0 & 0 \\
                0 & 0 & 0 & 1 & 0 & 0 & 0 & 0 \\
                0 & 0 & 0 & 0 & 1 & 0 & 0 & 0 \\
                0 & 0 & 0 & 0 & 0 & 0 & i & 0 \\
                0 & 0 & 0 & 0 & 0 & i & 0 & 0 \\
                0 & 0 & 0 & 0 & 0 & 0 & 0 & 1
            \end{bsmallmatrix} 
\\  \notag 
H_\Gate{CiSwap} &= \tfrac{\pi}{8}( Z_0 X_1 X_2 + Z_0 Y_1 Y_2 - X_1 X_2 - Y_1 Y_2)
\]
Can be decomposed into a 2 CNots and a doubly controlled-iX gate \eqref{CCiX}~\cite{gsc}.
$$
\adjustbox{scale=0.75}{\begin{quantikz}[thin lines, column sep=0.75em,row sep={2.5em,between origins}]
& \qw & \ctrl{2} & \qw \\
& \qw & \gate[2]{\text{iSwap}} & \qw \\
& \qw &  & \qw 
& \end{quantikz}}
\simeq
\adjustbox{scale=0.8}{\begin{quantikz}[thin lines, column sep=0.75em,row sep={2.5em,between origins}]
& \qw & \ctrl{1} & \qw & \qw \\
& \targ{} & \ctrl{1} & \targ{} & \qw \\
& \ctrl{-1} & \gate{iX} & \ctrl{-1} & \qw
\end{quantikz}
}
$$
Rasmussen and Zinner (2020)~\cite{Rasmussen2020a} discuss possible implementations using superconducting circuits.




 
\paragraph{Margolus gate}~\cite{Margolus1994a,Barenco1995b,DiVincenzo1998a,Song2003a,Maslov2016a,Linke2017a}
\index{Margolus gate}\index{simplified Toffoli gate|see{Margolus gate}}\index{relative phase Toffoli gate}
A``simplified'' Toffoli gate, that differs from the Toffoli only by a relative phase, in that the \ket{101} state picks up a $-1$ phase. In certain circuits Toffoli gates can be replaced with such relative phase Toffoli gates, leading to lower overall gate counts~\cite{Maslov2016a}.
 \[
        \Gate{Margolus} &= \begin{bsmallmatrix}
                1 & 0 & 0 & 0 & 0 & 0 & 0 & 0 \\
                0 & 1 & 0 & 0 & 0 & 0 & 0 & 0 \\
                0 & 0 & 1 & 0 & 0 & 0 & 0 & 0 \\
                0 & 0 & 0 & 1 & 0 & 0 & 0 & 0 \\
                0 & 0 & 0 & 0 & 1 & 0 & 0 & 0 \\
                0 & 0 & 0 & 0 & 0 & -1 & 0 & 0 \\
                0 & 0 & 0 & 0 & 0 & 0 & 0 & 1 \\
                0 & 0 & 0 & 0 & 0 & 0 & 1 & 0
            \end{bsmallmatrix} 
\\  \notag 
H_\Gate{Margolus} &= \tfrac{\pi}{8}(1-Z_0)(-2 -Z_1 X_2 + Z_1 Z_2 + X_2 + Z_2)
\]   
    
The Margolus gate is equivalent to a Toffoli gate plus a CCZ gate,    
$$
\adjustbox{scale=0.8}{\begin{quantikz}[thin lines, column sep=0.75em,row sep={2.5em,between origins}]
& \qw & \ctrl{1} & \qw & \ctrl{1} & \qw \\
& \gate{X} & \ctrl{1} & \gate{X} & \ctrl{1} & \qw \\
& \qw & \gate{Z} & \qw & \targ{} & \qw
\end{quantikz}
}
$$
 and can be implemented with only 3 \Gate{CNot} gates.
$$
\adjustbox{scale=0.8}{\begin{quantikz}[thin lines, column sep=0.75em,row sep={2.5em,between origins}]
& \qw & \qw & \qw & \qw & \qw & \qw & \qw & \ctrl{2} & \qw & \qw & \qw & \qw & \qw & \qw & \qw & \qw \\
& \qw & \qw & \qw & \ctrl{1} & \qw & \qw & \qw & \qw & \qw & \qw & \qw & \ctrl{1} & \qw & \qw & \qw & \qw \\
& \gate{V^\dagger} & \gate{T^\dagger} & \gate{V} & \targ{} & \gate{V^\dagger} & \gate{T^\dagger} & \gate{V} & \targ{} & \gate{V^\dagger} & \gate{T} & \gate{V} & \targ{} & \gate{V^\dagger} & \gate{T} & \gate{V} & \qw
\end{quantikz}
}
$$
Note that this decomposition is often expressed in terms of $R_y(\tfrac{\pi}{4})$, which is the same as $V T V^\dagger$ up to phase, e.g.~\citet[Ex~4.26]{Nielsen2000a}. This is a T-like gate: a counter-clockwise eighth turn of the Bloch sphere about the $\widehat{y}$-axis. \index{T-like gate}

