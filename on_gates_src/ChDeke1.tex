
% !TEX encoding = UTF-8 Unicode 
% !TEX root = on_gates.tex


\clearpage
\section{Decomposition of 1-qubit gates}
A general 1-qubit gate corresponds to some 2 by 2 unitary matrix,
\[
U = 
e^{i\alpha}\begin{bsmallmatrix}
a & -b^* \\
b & a^*
\end{bsmallmatrix}\]
where $a$ and $b$ are complex with $|a|^2 + |b|^2 = 1$, and  $\alpha$ is real.
Given such a generic unitary, we would like to represent this gate using standard parameterized gates.   

The first step to deke\footnote{
{\sl deke} {\sl |dek|} verb --- To decompile, deconstruct, or decompose.
\\ 1995  Neal Stephenson {\sl The Diamond Age} ``We gotta deke all this stuff now'' Easy come, easy go.
\index{deke}}\index{deke} a gate is to extract the phase factor $\alpha$,
\[
V = e^{i\alpha} U
\]
so that $V$ is a special unitary matrix with $\det V=1$. In general, if we multiply a special unitary matrix by a complex phase $c$ then $\det cU=c^k$ where $k$ is the rank of the matrix, i.e. $k=2^n$ for $n$ qubits. This follows since the determinate is the product of the eigenvalues, and multiplying a matrix by a constant multiplies each eigenvalue by that constant.
Thus the determinate of $U$ is $\det U=e^{i 2 \alpha }$, and we can extract the phase factor $\alpha$ with some trigonometry.
\[
\alpha = \half \operatorname{arctan2}\bigl(\Im(\det U), \Re(\det U)\bigr)
\label{phase_extraction}
\]
% TODO: CHECK

The two-argument arctangent function $\operatorname{arctan2}(y,x)$ returns the angle~$\theta$ between x-axis and the ray from the origin to $(x, y)$. In contrast the single argument arctangent function $\arctan(y/x)$ only gives the correct answer for $x>0$ since it can't distinguish between $(x,y)$ and $(-x, -y)$.\index{arctan2}
\begin{center}
\begin{tikzpicture}[scale=0.75]
\draw[->] (-3,0) -- (3, 0);
\draw[->] (0,-3) -- (0, 3);
\draw[-o, dashed] (0,0) --(2.5, 2) node[right] {$(x,y)$} node [midway, above, sloped] (r) {$r$};
\draw[->] (0:2) arc[radius=2, start angle=0, end angle=35] node [midway, right] {$\theta = \operatorname{arctan2}(y,x)$};
\end{tikzpicture}
\end{center}
For a complex number $x+iy = r e^{i\theta}$, the modulus (or magnitude) $r=\sqrt{x^2+y^2}$ and the phase (or argument) is $\theta = \operatorname{arctan2}(y,x)$. 

\subsection{Z-Y-Z decomposition}\index{ZYZ decomposition}
\label{sec:ZYZdeke}
\index{Pauli-rotation decomposition}
\todo{Naming. Pauli deke? Euler? Z-Y-Z or Z-Y?}
Any 1-qubit gate can be decomposed as a sequence of Z, Y, and Z rotations, and a phase~\cite{Barenco1995b}\footnote{The Z-Y decomposition is of ancient origin, long know in the theory of light polarization~\cite{Sousa2006a}}.
\[
U =  e^{i\alpha}\ R_z(\theta_2)\ R_y(\theta_1)\ R_z(\theta_0)
\]
Or in circuit notation.
\begin{center}
\adjustbox{scale=0.75}{\begin{quantikz}[thin lines, column sep=0.75em,row sep={2.5em,between origins}]
& \gate{U} & \qw 
\end{quantikz}}
$=$
\adjustbox{scale=0.8}{\begin{quantikz}[thin lines, column sep=0.75em,row sep={2.5em,between origins}]
& \gate{R_z(\theta_{0})} & \gate{R_y(\theta_{1})} & \gate{R_z(\theta_{2})} & \gate{\text{Ph}(\alpha)} & \qw
\end{quantikz}
}
\end{center}
Note that we have numbered the three angles in chronological order, and recall that time runs right-to-left in operator notation, but left-to-right in circuit notation. 

%%For example, the $R_x$ gate can be generated by an $R_y$ gate sandwiched between quarter turns about the z-axis.
%%\[
%%\Gate{R_x}(\theta) =  \ R_z(-\tfrac{\pi}{2})\ R_y(\theta)\ R_z(+\tfrac{\pi}{2})
%%\]
If multiply out the circuit, then we get the following universal 1-qubit gate.
\[
U = 
e^{i\alpha} \begin{bmatrix}
+e^{+i(+\half\theta_2 + \half\theta_0)} \cos(\half\theta_1) 
& -e^{-i(-\half\theta_2 + \half\theta_0)} \sin(\half\theta_1) 
\\
+e^{+i(+\half\theta_2 - \half\theta_0)} \sin(\half\theta_1) 
& +e^{-i(+\half\theta_2 +\half\theta_0)} \cos(\half\theta_1)
\end{bmatrix}
\]
The first step in the decomposition is to extract the phase using Eq.~\eqref{phase_extraction}, leaving a special unitary matrix $V = e^{-i\alpha} U$.
%
The value of $\theta_1$ can be calculated from the absolute value of either the diagonal or off-diagonal elements, provided those entries aren't close to zero.
For instance, the Z-gate has zero off-diagonal entries, whereas the X-gate has zeros on the diagonal. But the diagonal and off-diagonal entries can't approach zero at the same time.  So to calculate $\theta_1$ with greatest numerical accuracy, we use whichever element has the largest absolute value.
\[
\theta_1 = 
  \begin{cases}
		2 \arccos(|V_{00}|), & |V_{00}| \geq |V_{01}|\\	
		2 \arcsin(|V_{01}|), & |V_{00}| < |V_{01}|		
	\end{cases} 
\]

Having extracted $\theta_1$, we can now calculate the sum $\theta_0+\theta_1$ from $V_{11}$ using the $\operatorname{arctan2}$ function.\index{arctan2}
\[
 \theta_0 + \theta_2 =  
  2\ \operatorname{arctan2}\Left(\Im(\frac{V_{11}}{\cos(\half\theta_1)}),\ \Re(\frac{V_{11}}{\cos(\half\theta_1)})\Right)
\ .
\label{theta_sum}
\]
except if $\cos(\half\theta_1)=0$ then $\theta_0 + \theta_2 = 0$.



Similarly we can extract the difference  $\theta_0 - \theta_2$ from $V_{10}$.
\[
 \theta_0 - \theta_2 = 2\ \operatorname{arctan2}\Left(\Im(\frac{V_{10}}{\sin(\half\theta_1)}),\ \Re(\frac{V_{10}}{\sin(\half\theta_1)})\Right)
 \, 
 \label{theta_diff}
\]
again with an exception that if $\sin(\half\theta_1)=0$ then $\theta_0 - \theta_2 = 0$.
%
Taking the sum and differences of~\eqref{theta_sum} and~\eqref{theta_diff} yields $\theta_0$ and $\theta_2$, which completes the decomposition. 

Instead of rotation gates, we could express the same decomposition as Pauli-power gates with a reparameterization. 
\[
U = & e^{i\alpha'}\ Z^{t_2}\ Y^{t_1}\ Z^{t_0}
\\
& \alpha' = \alpha -(\theta_0+\theta_1+\theta_2)/\pi \notag \\
& t_0 = \theta_0/\pi \notag\\
& t_1 = \theta_1/\pi \notag\\
& t_2 = \theta_2/\pi \notag
\]
\index{Pauli-power gates! decomposition}

\subsection{V-Z decomposition}\index{V-Z decomposition}
For some superconducting qubit architectures the natural 1-qubit gates are Z-rotations $R_z$ and $V$ \eqref{V}\footnote{Note that here \Gate{V} refers to a specific 1-qubit gate, the square-root of the \Gate{X} gate, whereas elsewhere $V$ is used to denote a general unitary or special unitary matrix. Such notational ambiguities are inevitable since there's only so many squiggles to go around~\cite{???}.}  the square root of $X$~\cite{???}. There isn't direct access to $R_y$ rotations or general $R_x$ rotations, but this is only a minor inconvenience since $R_z(\theta) = V^\dagger R_y(\theta) V$, \secref{???} and we can therefore decompose 1-qubit gates to a 5-gate sequence,
\[
U =  e^{i\alpha}\ R_z(\theta_2)\ V^\dagger\ R_z(\theta_1)\ V\  R_z(\theta_0)  \ .
\]
\todo{Circuit}
\todo{Name of IBM architecture? Check location of dagger on Vs}




 


% ================================================================================
\subsection{General Euler angle decompositions}
% TODO: Check terminology, Euler decomposition
\index{Pauli-rotation decomposition}
Instead of a Z-Y-Z decomposition, we might instead desire a different decomposition, for example X-Y-X.
\[
U &=  R_x(\theta_2) R_y(\theta_1) R_x(\theta_0)
\]
The trick is to perform a similarity transform that takes us back to the Z-Y-Z decomposition that we already know how to perform. 
\[
V = C U C^\dagger &= C R_y(\theta_2)  C^\dagger\ C R_z(\theta_1)C^\dagger\ C R_y(\theta_0) C^\dagger
\\
& = R_z(\theta_2) R_y(\theta_1) R_z(\theta_0) \notag
\]
Here we want the single qubit gate $C$ that moves the $+\widehat{y}$ axis to $+\widehat{x}$, but leaves the $\widehat{z}$ axis alone. Consulting page~\pageref{S} we see that the required gate is $S^\dagger$. Therefore to find the parameters of a X-Y-X decomposition we carry out the similarity transform $V=S^\dagger\ U \ S$ and then perform a Z-Y-Z decomposition. 

There are 6 distinct proper-Euler decompositions, and the appropriate similarity transforms to Z-Y-Z are listed in Table~\ref{tab:geneuler}. These are all 1-qubit Clifford gates (Table~\ref{tab:Clifford1q}).   

\begin{table}[tp]
\caption{Euler decompositions}
\label{tab:geneuler}
\begin{center}
\begin{tabular}{cll}
~~~ Euler decomposition ~~~ & Similarity transform to Z-Y-Z \\
    X-Y-X & $h^\dagger$ \\
    X-Z-X & $C$ & %$R_n(+\tfrac{2\pi}{3}, \tfrac{1}{\sqrt{3}}, \tfrac{1}{\sqrt{3}}, \tfrac{1}{\sqrt{3}})$
     \\
    Y-X-Y & $C^\dagger$ & %$R_n(-\tfrac{2\pi}{3}, \tfrac{1}{\sqrt{3}}, \tfrac{1}{\sqrt{3}}, \tfrac{1}{\sqrt{3}})$
     \\
    Y-Z-Y & $V H V^\dagger$ & %$R_n(\pi, 0, \tfrac{1}{\sqrt{2}}, \tfrac{1}{\sqrt{2}})$ 
    \\
    Z-X-Z & $S^\dagger$ \\
    Z-Y-Z & $I$
\end{tabular}
\end{center}
\end{table}


\subsection{Bloch rotation decomposition}
\index{Bloch rotation decomposition}
Finally lets consider the decompositions of 1-qubit gates into single rotations about a particular axis~\eqref{Rn}.
\[
R_{\vec{n}}(\theta) =
\begin{bmatrix*}
	\cos(\half\theta) - i n_z \sin(\half\theta)  &
	- n_y \sin(\half\theta)-i n_x \sin(\half\theta)  \\
	n_y \sin(\half\theta)-i n_x \sin(\half\theta)   & 
	\cos(\half\theta) + i n_z \sin(\half\theta)
\end{bmatrix*}
\]
Assuming that we have already extracted the phase and therefore $V$ is a 1-qubit special unitary matrix, we can proceed as follows.
\[
	N &= \sqrt{(\Im V_{0,1})^2 + (\Re V_{0,1})^2 + (\Im V_{0,0})^2} \\
	n_x & = -\Im V_{0,1}/N  \notag \\
	n_y & = - \Re V_{0,1}/N \notag \\
	n_z & = - \Im V_{0,0}/N \notag \\
s = \sin(\half\theta) &= - \Im V_{0, 0} / n_z \notag \\ 
c = \cos(\half\theta)  &= \Re V_{0, 0}  \notag \\
\theta &= 2\ \text{arctan2}(s, c) \notag
\]
The one ambiguous edge case that needs to be accounted for is that the identity can be represented as a zero-radians rotation about any axis. 


\subsection{Decomposition of Bloch rotation}
A rotation about an arbitrary axis in the Bloch sphere can be analytically decomposed into a sequence of five $R_z$ and $R_y$ gates~\cite{Glendinning2010a}.
\[
R_{\vec{n}}(\theta) & = R_z(+\alpha) R_y(+\beta) R_z(\theta) R_y(-\beta) R_z(-\alpha) \\
& \alpha = \operatorname{arctan2}(n_y, n_x) \notag \\
& \beta = \operatorname{arccos}(n_z) \notag 
\]
%[TODO: Explanation]
%[TODO: Turn into circuit]



