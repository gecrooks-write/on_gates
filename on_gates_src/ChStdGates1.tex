
% !TEX encoding = UTF-8 Unicode 
% !TEX root = on_gates.tex



\clearpage
\section{Standard 1-qubit gates}
% TODO: Define "standard" gates


Classically, there are only 2 1-bit reversible logic gates, identity and NOT (And 2 irreversible gates, reset to 0 and reset to 1). But in quantum mechanics the zero and one states can be placed into superposition, so there are many other interesting possibilities. 

\subsection{Pauli gates}
\index{Pauli gates}
\index{Pauli gates!commutation relations}
The simplest 1-qubit gates are the 4 gates represented by the Pauli operators, I, X, Y, and Z. These operators are also sometimes notated as $\sigma_x$, $\sigma_y$, $\sigma_z$, or with an index $\sigma_i$, so that $\sigma_0=I$, $\sigma_1=X$, $\sigma_2=Y$, $\sigma_3=Z$. % Move to section on pauli group
\index{$\sigma_x$}\index{$\sigma_y$}\index{$\sigma_z$}

We will explore the algebra of Pauli operators in more detail in chapter~\secref{sec:Clifford}. But for now, note that the Pauli gates are all Hermitian, $\sigma_i^\dagger=\sigma_i$, square to the identity $\sigma_i^2 =I$, and that the $X$, $Y$, and $Z$ gates anti-commute with each other.
\[
XY = -YZ = iZ \notag \\
YZ = -ZY = iX \notag \\
ZX = -ZX = iY \notag \\
XYZ = iI \notag
\]


\paragraph{Pauli-I gate} (identity):\index{identity gate! 1-qubit}
\index{I gate@\Gate{I} gate|see {identity gate}}
\[
I = \begin{bmatrix}1 & 0 \\ 0 & 1 \end{bmatrix}
\]
\begin{center}
\adjustbox{scale=0.8}{\begin{quantikz}[thin lines, column sep=0.75em,row sep={2.5em,between origins}]
& \gate{I} & \qw
\end{quantikz}
}
\end{center}
The trivial no-operation gate on 1-qubit, represented by the identity matrix. Acting on any arbitrary state, the gate leave the state unchanged.%
%
%\begin{center}
%\adjustbox{scale=0.75}{
% \begin{tikzpicture}[scale=1.5]
%   \begin{scope}[canvas is zy plane at x=0]
%     \draw (0,0) circle (1cm);
%     %\draw[ultra thin] (-1,0) -- (1,0) (0,-1) -- (0,1);
%     \draw[->] (0,0) -- (1.35,0) node[below] {$\widehat{x}$};
%%     \draw[dashed] (0,0) -- (1.,1.) ;
%   \end{scope}
%
%   \begin{scope}[canvas is zx plane at y=0]
%     \draw (0,0) circle (1cm);
%     %\draw (-1,0) -- (1,0) (0,-1) -- (0,1);
%     \draw[->] (0,0) -- (0,1.175) node[right] {$\widehat{y}$};
%   \end{scope}
%
%   \begin{scope}[canvas is xy plane at z=0]
%     \draw (0,0) circle (1cm);
%	%\draw (-1,0) -- (1,0) (0,-1) -- (0,1);
%	\draw[->] (0,0) -- (0,1.175) node[above] {$\widehat{z}$};
%   \end{scope}
%
%%   \begin{scope}[canvas is zx plane at y=1.0]	
%%   	\centerarc[blue,->](0,0)(270:90:0.25)
%%   \end{scope}
%    
%   \draw[->] (1.5,0) -- node[above] {$I$} ++(0.5, 0) ;
%
%	\begin{scope}[xshift=3.5cm]
%    \begin{scope}[canvas is zy plane at x=0]
%     \draw (0,0) circle (1cm);
%     %\draw[ultra thin] (-1,0) -- (1,0) (0,-1) -- (0,1);
%     \draw[->] (0,0) -- (1.35,0) node[below] {$\widehat{x}$};
%%     \draw[dashed] (0,0) -- (1.,1.) ;
%   \end{scope}
%
%   \begin{scope}[canvas is zx plane at y=0]
%     \draw (0,0) circle (1cm);
%     %\draw (-1,0) -- (1,0) (0,-1) -- (0,1);
%     \draw[->] (0,0) -- (0,1.175) node[right] {$\widehat{y}$};
%   \end{scope}
%
%   \begin{scope}[canvas is xy plane at z=0]
%     \draw (0,0) circle (1cm);
%	%\draw (-1,0) -- (1,0) (0,-1) -- (0,1);
%	\draw[->] (0,0) -- (0,1.175) node[above] {$\widehat{z}$};
%   \end{scope}
%
%\end{scope}
% \end{tikzpicture}
%}
% \end{center}
%
%
\[
I &=\ket{0}\!\bra{0} + \ket{1}\!\bra{1} \notag \\
I&\ket{0} = \ket{0} \notag \\ 
I&\ket{1} = \ket{1} \notag
\]




\paragraph{Pauli-X gate} (X gate, bit flip) %, NOT, bit flip, negation) \cite{Barenco1995b}
\index{Pauli-X gate}
\index{X@$\Gate{X}$|see {Pauli-X gate}}
\index{bit flip|see {Pauli-X gate}}
\[
X = \begin{bmatrix}0 & 1 \\ 1 & 0 \end{bmatrix}
\]
\begin{center}
\adjustbox{scale=0.8}{\begin{quantikz}[thin lines, column sep=0.75em,row sep={2.5em,between origins}]
& \gate{X} & \qw
\end{quantikz}
}
\end{center}


The $X$-gate generates a half-turn in the Bloch sphere about the $x$ axis. 
%
\begin{center}
\adjustbox{scale=0.75}{
 \begin{tikzpicture}[scale=1.5]
   \begin{scope}[canvas is zy plane at x=0]
     \draw (0,0) circle (1cm);
     %\draw[ultra thin] (-1,0) -- (1,0) (0,-1) -- (0,1);
     \draw[->] (0,0) -- (1.35,0) node[below] {$\widehat{x}$};
%     \draw[dashed] (0,0) -- (1.,1.) ;
   \end{scope}

   \begin{scope}[canvas is zx plane at y=0]
     \draw (0,0) circle (1cm);
     %\draw (-1,0) -- (1,0) (0,-1) -- (0,1);
     \draw[->] (0,0) -- (0,1.175) node[right] {$\widehat{y}$};
   \end{scope}

   \begin{scope}[canvas is xy plane at z=0]
     \draw (0,0) circle (1cm);
	%\draw (-1,0) -- (1,0) (0,-1) -- (0,1);
	\draw[->] (0,0) -- (0,1.175) node[above] {$\widehat{z}$};
   \end{scope}

%   \begin{scope}[canvas is zx plane at y=1.0]	
%   	\centerarc[blue,->](0,0)(270:90:0.25)
%   \end{scope}
    
   \draw[->] (1.5,0) -- node[above] {$X$} ++(0.5, 0) ;

	\begin{scope}[xshift=3.5cm]
    \begin{scope}[canvas is zy plane at x=0]
     \draw (0,0) circle (1cm);
     %\draw[ultra thin] (-1,0) -- (1,0) (0,-1) -- (0,1);
     \draw[->] (0,0) -- (1.35,0) node[below] {$\widehat{x}$};
%     \draw[dashed] (0,0) -- (1.,1.) ;
   \end{scope}

   \begin{scope}[canvas is zx plane at y=0]
     \draw (0,0) circle (1cm);
     %\draw (-1,0) -- (1,0) (0,-1) -- (0,1);
     \draw[->] (0,0) -- (0,-1.175) node[left] {$\widehat{y}$};
   \end{scope}

   \begin{scope}[canvas is xy plane at z=0]
     \draw (0,0) circle (1cm);
	%\draw (-1,0) -- (1,0) (0,-1) -- (0,1);
	\draw[->] (0,0) -- (0,-1.175) node[below] {$\widehat{z}$};
   \end{scope}

\end{scope}
 \end{tikzpicture}
}
 \end{center}


With respect to the computational basis, the $X$ gate is equivalent to a classical NOT operation, or logical negation. The computation basis states are interchanged, so that $\ket{0}$ becomes $\ket{1}$ and $\ket{1}$ becomes $\ket{0}$.
\index{NOT gate}\index{logical negation}
\[
X &=\ket{1}\!\bra{0} + \ket{0}\!\bra{1} \notag \\
X&\ket{0} = \ket{1} \notag \\ 
X&\ket{1} = \ket{0} \notag
\]
However, the X-gate is not a true quantum NOT gate, since it only logically negates the state in the computational basis. A true quantum logical negation would require mapping every point on the Bloch sphere to its antipodal point. But that would require an inversion of the sphere which cannot be generated by rotations alone. There is no general quantum NOT operation that would negate an arbitrary qubit state.


\paragraph{Pauli-Y gate} (Y-gate):
\index{Pauli-Y gate}
\index{Y gate@\Gate{Y} gate|see {Pauli-Y gate}}
\[
Y = \begin{bmatrix*}[r]0 & -i \\ i & 0 \end{bmatrix*}
\]
\begin{center}
\adjustbox{scale=0.8}{\begin{quantikz}[thin lines, column sep=0.75em,row sep={2.5em,between origins}]
& \gate{Y} & \qw
\end{quantikz}
}
\end{center}
A useful mnemonic for remembering where to place the minus sign in the matrix of the Y gate is ``Minus eye high''~\cite{???}. \index{minus eye high}
% TODO: What's the origin of this?
In some older literature the Y-gate is defined as $iY=\begin{bsmallmatrix} 0 & 1 \\ -1 & 0 \end{bsmallmatrix}$ (e.g.~\cite{Rieffel2014a}), which is the same gate up to a phase.


The Pauli-Y gate generates a half-turn in the Bloch sphere about the $\widehat{y}$ axis.
\begin{center}
\adjustbox{scale=0.75}{
 \begin{tikzpicture}[scale=1.5]
   \begin{scope}[canvas is zy plane at x=0]
     \draw (0,0) circle (1cm);
     %\draw[ultra thin] (-1,0) -- (1,0) (0,-1) -- (0,1);
     \draw[->] (0,0) -- (1.35,0) node[below] {$\widehat{x}$};
%     \draw[dashed] (0,0) -- (1.,1.) ;
   \end{scope}

   \begin{scope}[canvas is zx plane at y=0]
     \draw (0,0) circle (1cm);
     %\draw (-1,0) -- (1,0) (0,-1) -- (0,1);
     \draw[->] (0,0) -- (0,1.175) node[right] {$\widehat{y}$};
   \end{scope}

   \begin{scope}[canvas is xy plane at z=0]
     \draw (0,0) circle (1cm);
	%\draw (-1,0) -- (1,0) (0,-1) -- (0,1);
	\draw[->] (0,0) -- (0,1.175) node[above] {$\widehat{z}$};
   \end{scope}

%   \begin{scope}[canvas is zx plane at y=1.0]	
%   	\centerarc[blue,->](0,0)(270:90:0.25)
%   \end{scope}
    
   \draw[->] (1.5,0) -- node[above] {$Y$} ++(0.5, 0) ;

	\begin{scope}[xshift=3.5cm]
    \begin{scope}[canvas is zy plane at x=0]
     \draw (0,0) circle (1cm);
     %\draw[ultra thin] (-1,0) -- (1,0) (0,-1) -- (0,1);
     \draw[->] (0,0) -- (-1.35,0) node[above] {$\widehat{x}$};
%     \draw[dashed] (0,0) -- (1.,1.) ;
   \end{scope}

   \begin{scope}[canvas is zx plane at y=0]
     \draw (0,0) circle (1cm);
     %\draw (-1,0) -- (1,0) (0,-1) -- (0,1);
     \draw[->] (0,0) -- (0,1.175) node[right] {$\widehat{y}$};
   \end{scope}

   \begin{scope}[canvas is xy plane at z=0]
     \draw (0,0) circle (1cm);
	%\draw (-1,0) -- (1,0) (0,-1) -- (0,1);
	\draw[->] (0,0) -- (0,-1.175) node[below] {$\widehat{z}$};
   \end{scope}

\end{scope}
 \end{tikzpicture}
}
 \end{center}

The $Y$-gate can be thought of as a combination of $X$ and $Z$ gates, $Y=-iZX$. With respect to the computational basis, we interchange the zero and one states and apply a relative phase flip.
\[
Y & =i\ket{1}\!\bra{0} -i \ket{0}\!\bra{1} \notag\\
Y & \ket{0} = +i\ket{1} \notag \\ 
Y & \ket{1} = -i\ket{0} \notag
\]




\paragraph{Pauli-Z gate} (Z-gate, phase flip)%
\index{Pauli-Z gate}%
\index{phase flip|see {Pauli-Z gate}}%
\index{Z gate|see {Pauli-Z gate}}
\[
Z = \begin{bmatrix*}[r]1 & 0 \\ 0 & -1 \end{bmatrix*}
\]
\begin{center}
\adjustbox{scale=0.8}{\begin{quantikz}[thin lines, column sep=0.75em,row sep={2.5em,between origins}]
& \gate{Z} & \qw
\end{quantikz}
}
\end{center}
\[
H_Z = - \pi\half(1 - Z)
\notag
\]


The Pauli-Z gate generates a half-turn in the Bloch sphere about the $\widehat{z}$ axis.
\begin{center}
\adjustbox{scale=0.75}{
 \begin{tikzpicture}[scale=1.5]
   \begin{scope}[canvas is zy plane at x=0]
     \draw (0,0) circle (1cm);
     %\draw[ultra thin] (-1,0) -- (1,0) (0,-1) -- (0,1);
     \draw[->] (0,0) -- (1.35,0) node[below] {$\widehat{x}$};
%     \draw[dashed] (0,0) -- (1.,1.) ;
   \end{scope}

   \begin{scope}[canvas is zx plane at y=0]
     \draw (0,0) circle (1cm);
     %\draw (-1,0) -- (1,0) (0,-1) -- (0,1);
     \draw[->] (0,0) -- (0,1.175) node[right] {$\widehat{y}$};
   \end{scope}

   \begin{scope}[canvas is xy plane at z=0]
     \draw (0,0) circle (1cm);
	%\draw (-1,0) -- (1,0) (0,-1) -- (0,1);
	\draw[->] (0,0) -- (0,1.175) node[above] {$\widehat{z}$};
   \end{scope}

%   \begin{scope}[canvas is zx plane at y=1.0]	
%   	\centerarc[blue,->](0,0)(270:90:0.25)
%   \end{scope}
    
   \draw[->] (1.5,0) -- node[above] {$Z$} ++(0.5, 0) ;

	\begin{scope}[xshift=3.5cm]
   \begin{scope}[canvas is zy plane at x=0]
     \draw (0,0) circle (1cm);
     %\draw[ultra thin] (-1,0) -- (1,0) (0,-1) -- (0,1);
     \draw[->] (0,0) -- (-1.35,0) node[above] {$\widehat{x}$};
   \end{scope}

   \begin{scope}[canvas is zx plane at y=0]
     \draw (0,0) circle (1cm);
     %\draw (-1,0) -- (1,0) (0,-1) -- (0,1);
     \draw[->] (0,0) -- (0,-1.175) node[left] {$\widehat{y}$};
   \end{scope}

   \begin{scope}[canvas is xy plane at z=0]
     \draw (0,0) circle (1cm);
	%\draw (-1,0) -- (1,0) (0,-1) -- (0,1);
	\draw[->] (0,0) -- (0,1.175) node[above] {$\widehat{z}$};
   \end{scope}
	\end{scope}


 \end{tikzpicture}
}
 \end{center}

With respect to the computational basis, the $Z$ gate flips the phase of the $\ket{1}$ state relative to the $\ket{0}$ state.
\[
Z &=\ket{0}\!\bra{0} - \ket{1}\!\bra{1} \notag\\
Z& \ket{0}  = +\ket{0} \notag \\ 
Z&\ket{1} = -\ket{1} \notag
\]

\todo{ Comment on Z gate}

% Pauli gates are all Hermitian



\subsection{Rotation gates}\index{Rotation gates}
The three Pauli-rotation gates\footnote{The 1-qubit rotation gates are typically verbalized as {\sl arr-ex}, {\sl arr-why}, {\sl arr-zee}, and {\sl arr-en}.\index{pirate gates}} $R_x$, $R_y$, and $R_z$ rotate the state vector by an arbitrary angle about the corresponding axis in the Bloch sphere, Fig.~\ref{fig:paulirotations}. They are generated by taking exponentials of the Pauli operators.
\todo{Wordsmith} 

A useful identity to keep in mind is that given an operator~$A$ that squares to the identity $A^2=I$ then
\[
\exp(i\theta A) = \cos(\theta)\ I + i \sin(\theta)\ A
\]
This is a generalization of the usual Euler's formula $e^{ix} =\cos x + i \sin x$. We expand the exponential as a power series, and gather the even powers into the cosine term, and the odd powers into the sin term.
\[
\exp(i\theta A) & = I + i \theta A - \tfrac{\theta^2}{2!} I - i \tfrac{\theta^3}{3!} A 
- \tfrac{\theta^4}{4!} I - i \tfrac{\theta^5}{5!} A +
+ \cdots \notag \\
& = \left(1- \tfrac{\theta^2}{2!} +\tfrac{\theta^4}{4!} - \cdots\right) I +  
    \left(\theta -\tfrac{\theta^3}{3!} +\tfrac{\theta^5}{5!}- \cdots\right)A
    \notag
    \\ \notag
 & = \cos(\theta)\ I + i \sin(\theta)\ A
\]


\paragraph{$R_x$ gate}\cite{Barenco1995b}\index{Rx gate@\Gate{R_x} gate} Rotate $\theta$ radians anti-clockwise about the \axis{x} axis of the Bloch sphere.
\[
        R_x(\theta) & =  e^{-i \half\theta X} 
        			\\ \notag & = \cos(\half\theta) I -i \sin(\half\theta) X		
\\ \notag
			 &= \begin{bmatrix*}[r]
                            \cos(\half\theta) & -i \sin(\half\theta) \\
                            -i \sin(\half\theta) & \cos(\half\theta)
                        \end{bmatrix*}                        
\\ \notag
    H_{R_x} & = \half\theta X 
    \]

The $R_x$ gate is represented by the following circuit diagram.
$$
\adjustbox{scale=0.8}{\begin{quantikz}[thin lines, column sep=0.75em,row sep={2.5em,between origins}]
& \gate{R_x(\theta)} & \qw
\end{quantikz}
}
$$
or, if we want to specify a generic $R_x$ gate, and not a specific angle, we can drop the theta argument.
$$
\adjustbox{scale=0.8}{\begin{quantikz}[thin lines, column sep=0.75em,row sep={2.5em,between origins}]
& \gate{R_x} & \qw
\end{quantikz}
}
$$

\paragraph{$R_y$ gate}\cite{Barenco1995b}\index{Ry gate@\Gate{R_y} gate} Rotate $\theta$ radians anti-clockwise about the $\widehat{y}$ axis of the Bloch sphere.
\[
        R_y(\theta) 
        & =  e^{-i \half\theta Y} 
        			\\ \notag &  = \cos(\half\theta) I -i \sin(\half\theta) Y		
\\ \notag
			 &=
           \begin{bmatrix*}[r]
 			\cos(\half\theta) & -\sin(\half\theta)
        	\\ \sin(\half\theta) & \cos(\half\theta)
                        \end{bmatrix*}
\]
$$
\adjustbox{scale=0.8}{\begin{quantikz}[thin lines, column sep=0.75em,row sep={2.5em,between origins}]
& \gate{R_y(\theta)} & \qw
\end{quantikz}
}
$$
    \[
    \notag
    H_{R_y} & = \half\theta Y
    \]




\paragraph{$R_z$ gate}\cite{Barenco1995b}\index{Rz gate@\Gate{R_z} gate}
Rotate $\theta$ radians anti-clockwise about the $\widehat{z}$ axis of the Bloch sphere.

\[
        R_z(\theta) 
        & =  e^{-i \half\theta Z} 
        			\\ \notag & = \cos(\half\theta) I -i \sin(\half\theta) Z		
\\ \notag
			 &=
           \begin{bmatrix*}
        e^{-i\half\theta} & 0 \\
        0 & e^{+i\half\theta}
                        \end{bmatrix*}
\]
$$
\adjustbox{scale=0.8}{\begin{quantikz}[thin lines, column sep=0.75em,row sep={2.5em,between origins}]
& \gate{R_z(\theta)} & \qw
\end{quantikz}
}
$$
    \[
    \notag
    H_{R_z} & = \half\theta Z 
    \]

% TODO: Comment on being diagonal?


Consecutive rotations about the same axis can be merged, with the total angle being the sum of angles.
$$
\adjustbox{scale=0.8}{\begin{quantikz}[thin lines, column sep=0.75em,row sep={2.5em,between origins}]
& \gate{R_x(\theta_{0})} & \gate{R_x(\theta_{1})} & \qw
\end{quantikz}
} = \adjustbox{scale=0.8}{\begin{quantikz}[thin lines, column sep=0.75em,row sep={2.5em,between origins}]
& \gate{R_x(\theta_{0}+\theta_{1})} & \qw
\end{quantikz}
}
$$
$$
\adjustbox{scale=0.8}{\begin{quantikz}[thin lines, column sep=0.75em,row sep={2.5em,between origins}]
& \gate{R_y(\theta_{0})} & \gate{R_y(\theta_{1})} & \qw
\end{quantikz}
} = \adjustbox{scale=0.8}{\begin{quantikz}[thin lines, column sep=0.75em,row sep={2.5em,between origins}]
& \gate{R_y(\theta_{0}+\theta_{1})} & \qw
\end{quantikz}
}
$$
$$
\adjustbox{scale=0.8}{\begin{quantikz}[thin lines, column sep=0.75em,row sep={2.5em,between origins}]
& \gate{R_z(\theta_{0})} & \gate{R_z(\theta_{1})} & \qw
\end{quantikz}
} = \input{circuits/Rz01}
$$



Let us demonstrate that the $R_z$ gate generates rotations about the $\widehat{z}$ axis. Recall the definition of the Bloch vector of an arbitrary state $\ket{\psi}$, \secref{sec:blochsphere}.
\[
R_z(\theta') \ket{\psi} &= 
	\Bigl(e^{-i\half\theta'} \ket{0}\!\bra{0} + e^{+i\half\theta'} \ket{1}\!\bra{1}\Bigr)
	\Bigl( \cos(\half\theta)\ket{0} +e^{i\phi}\sin(\half\theta)\ket{1}\Bigr)
\notag
\\
\notag 
& = e^{-i\half\theta'} \Bigl( cos(\half\theta)\ket{0} + e^{i(\theta'+\phi)} \ket{1}\Bigr)
\\ 
& \simeq cos(\half\theta)\ket{0} + e^{i(\theta'+\phi)}\ket{1}
\]
In the last line we drop an irrelevant phase. We can see that the $R_z$ gate has left the elevation angle unchanged, but added $\theta'$ to the azimuth angle, which corresponds to a rotation about the \axis{z}-axis. 

We can do the same exercise for the $R_X$ and $R_Y$ gates, although the trigonometry is slightly more involved.

%TODO: R_X or R_x?

\begin{figure}[t]
\begin{center}
%\adjustbox{scale=0.75}{
 \begin{tikzpicture}[scale=3]
   \begin{scope}[canvas is zy plane at x=0]
     \draw (0,0) circle (1cm);
     %\draw[ultra thin] (-1,0) -- (1,0) (0,-1) -- (0,1);
     \draw[->] (0,0) -- (1.2,0); 
     \draw (1.5, 0.5) node {$R_x(\theta)$};
   \end{scope}

   \begin{scope}[canvas is zx plane at y=0]
     \draw (0,0) circle (1cm);
     %\draw (-1,0) -- (1,0) (0,-1) -- (0,1);
     \draw[->] (0,0) -- (0,1.2) node[right] {$R_y(\theta)$};
   \end{scope}

   \begin{scope}[canvas is xy plane at z=0]
     \draw (0,0) circle (1cm);
	\draw[->] (0,0) -- (0,1.2) node[above] {$R_z(\theta)$};
	\end{scope}
		 
   \begin{scope}[canvas is xy plane at z=1.0]	
   	\centerarc[thick,blue, ->](0,0)(-135:450:0.25)
   \end{scope}

   \begin{scope}[canvas is zy plane at x=1.0]	
   	\centerarc[thick,blue, ->](0,0)(450:-135:0.25)
   \end{scope}

   \begin{scope}[canvas is zx plane at y=1.00]	
   	\centerarc[thick,blue, ->](0,0)(-135:450:0.25)
   \end{scope}
 \end{tikzpicture}
%    }
 \end{center}
\caption{Pauli rotations of the Bloch Sphere}
\label{fig:paulirotations}
\end{figure}



\paragraph{$R_{\vec{n}}$ gate} \index{Rn gate@$R_{\vec{n}}$ gate|see {rotation gate}}\index{rotation gate}
A rotation of $\theta$ radians anti-clockwise about an arbitrary axis in the Bloch sphere.
\[
\label{Rn}
R_{\vec{n}}(\theta) &
=  e^{-i\half \theta (n_x X+ n_y Y + n_z Z)}
\\ 
\notag
&= \cos(\half\theta) I - i \sin(\half\theta)(n_x X+ n_y Y + n_z Z)
\\
& = \notag
\begin{bmatrix*}
	\cos(\half\theta) - i n_z \sin(\half\theta)  &
	- n_y \sin(\half\theta)-i n_x \sin(\half\theta)  \\
	n_y \sin(\half\theta)-i n_x \sin(\half\theta)   & 
	\cos(\half\theta) + i n_z \sin(\half\theta)
\end{bmatrix*}
\]
\begin{center}
\adjustbox{scale=0.9}{\begin{quantikz}[thin lines, column sep=0.75em, row sep={2.5em,between origins}]
& \gate{R_{\vec{n}}(\theta)} & \qw
\end{quantikz}
}
\end{center}
Every 1-qubit gate can be represented as a rotation gate (up to phase) 
with some coordinate $(\theta\ n_x, \theta\ n_y, \theta\ n_z)$, where $n_x^2 +n_y^2 +n_z^2 =1$ and $\theta$ runs between $\pi$ and $-\pi$.
The Pauli gates are the rotations around the principal axes.
\[
R_x(\theta) & = R_{\vec{n}}(\theta), \quad \vec{n} = (1, 0, 0) \notag \\
R_y(\theta) & = R_{\vec{n}}(\theta), \quad \vec{n} = (0, 1, 0) \notag \\
R_z(\theta) & = R_{\vec{n}}(\theta), \quad \vec{n} = (0, 0, 1) \notag
\]
This representation provides a convenient visualization of 1-qubit gates: The 1-qubit gates form a spherical ball of radius $\theta$.  See figures~\ref{fig:sphere-of-gates} and~\ref{fig:GateCoords}. This sphere-of-gates is distinct from the Bloch sphere of states, although the underlying mathematical structures are related.

% TODO: Merging rotations


You might reasonably be wondering why there is a factor of half in the definitions of the rotation gates. A 1-qubit gate is represented by an element of the group $SU(2)$ (the group of $2\times2$ unitary matrices with unit determinate). Each element is a rotation in a 2-dimensional complex vector space. But we are visualizing the effect of these gates as rotations in 3-dimensional Euclidean space, which are elements of the special orthogonal group $SO(3)$. We can do this because there is an accidental correspondence between these two groups that allows us to visualize 1-qubit gates as rotations in 3-space. We can map two elements of $SU(2)$ (differing by only a -1 phase) to each element of $SO(3)$ while keeping the group structure. In the jargon, $SU(2)$ is a double cover of $SO(3)$. Because of this doubling up, a rotation of $\theta$ radians in the Bloch sphere corresponds to a rotation of only $\half \theta$ in the complex vector space. We have to go twice around the Bloch sphere, $\theta =4\pi$, to get back to the same gate with the same phase.


\todo{Check double cover nomenclature}

% Hamiltonian of R_n?



%
%\begin{figure}[t]
%\begin{center}
% \begin{tikzpicture}[scale=3]
%   \begin{scope}[canvas is zy plane at x=0]
%     \draw (0,0) circle (1cm);
%     %\draw[ultra thin] (-1,0) -- (1,0) (0,-1) -- (0,1);
%     \draw[] (0,0) -- (1.2,0) node[below left] {$R_x(\theta)$};
%   \end{scope}
%
%   \begin{scope}[canvas is zx plane at y=0]
%     \draw (0,0) circle (1cm);
%     %\draw (-1,0) -- (1,0) (0,-1) -- (0,1);
%     \draw[] (0,0) -- (0,1.1) node[right] {$R_y(\theta)$};
%   \end{scope}
%
%   \begin{scope}[canvas is xy plane at z=0]
%     \draw (0,0) circle (1cm);
%	\draw[] (0,0) -- (0,1.1) node[above] {$R_z(\theta)$};
%	\end{scope}
%		 
%   \begin{scope}[canvas is xy plane at z=1.0]	
%   	\centerarc[red,->](0,0)(450:135:0.25)
%   \end{scope}
%
%   \begin{scope}[canvas is zy plane at x=1.0]	
%   	\centerarc[blue,<-](0,0)(450:135:0.25)
%   \end{scope}
%
%   \begin{scope}[canvas is zx plane at y=1.0]	
%   	\centerarc[green,->](0,0)(450:135:0.25)
%   \end{scope}
%
%	 
%	 
%	 
% \end{tikzpicture}
% \end{center}
%\caption{Rotations of the Bloch Sphere}
%\end{figure}
%

\begin{figure}[tp]
\begin{center}
 \begin{tikzpicture}[scale=3]
   \begin{scope}[canvas is zy plane at x=0]
     \draw (0,0) circle (1cm);
     %\draw[ultra thin] (-1,0) -- (1,0) (0,-1) -- (0,1);
     \draw[->] (-1,0) -- (1.2,0) node[below left] {$\theta\ n_x$};
   \end{scope}

   \begin{scope}[canvas is zx plane at y=0]
     \draw (0,0) circle (1cm);
     %\draw (-1,0) -- (1,0) (0,-1) -- (0,1);
     \draw[->] (0,-1) -- (0,1.1) node[right] {$\theta\ n_y$};
   \end{scope}

   \begin{scope}[canvas is xy plane at z=0]
     \draw (0,0) circle (1cm);
	%\draw (-1,0) -- (1,0) (0,-1) -- (0,1);
	\draw[->] (0,-1) -- (0,1.1) node[above] {$\theta\ n_z$};
   \end{scope}

	 
 \end{tikzpicture}
 \end{center}

\caption[Spherical ball of 1-qubit gates]{Spherical ball of 1-qubit gates \eqref{Rn}. Each point within this sphere represents a unique 1-qubit gate (up to phase). 
Antipodal points on the surface represent the same gate. The Pauli rotation gates lie along the three principal axes.}
 \label{fig:sphere-of-gates} 
%\end{figure}

%\begin{figure}[t]
\begin{center}
 \begin{tikzpicture}[scale=3]
   \begin{scope}[canvas is zy plane at x=0]
     \draw (0,0) circle (1cm);
     \draw (-1,0) -- (1,0) (0,-1) -- (0,1);
   \end{scope}

   \begin{scope}[canvas is zx plane at y=0]
     \draw (0,0) circle (1cm);
     \draw (-1,0) -- (1,0) (0,-1) -- (0,1);
   \end{scope}

   \begin{scope}[canvas is xy plane at z=0]
     \draw (0,0) circle (1cm);
     \draw (-1,0) -- (1,0) (0,-1) -- (0,1);
   \end{scope}

	\node[fill=white] at (0,0,0) {$I$};

	\node[fill=white] at (0,1,0) {$Z$};
	\node[fill=white] at (0,0.5,0) {$S$};
	\node[fill=white] at (0,0.25,0) {$T$};			
	\node[fill=white] at (0,-0.5,0) {${S^\dagger}$};
	\node[fill=white] at (0,-0.25,0) {$T^\dagger$};	
	% \node[fill=white] at (0,-1,0) {Z};
		
	\node[fill=white] at (0,0,1) {$X$};
	\node[fill=white] at (0,0,0.5) {$V$};
	\node[fill=white] at (0,0,-0.5) {$V^\dagger$};	
	% \node[fill=white] at (0,0,-1) {X};
	
	\node[fill=white] at (0,{sqrt(1/2)},{sqrt(1/2)}) {$H$};
%	\node[fill=white] at (0,0,1) {H};
	
	\node[fill=white] at (1,0,0) {$Y$};
	\node[fill=white] at (0.5,0,0) {${h^\dagger}$};	
	\node[fill=white] at (-0.5,0,0) {${h}$};	
	% \node[fill=white] at (-1,0,0) {$Y$};
 \end{tikzpicture}
 \end{center}

\caption[Coordinates of common 1-qubit gates.]{Coordinates of common 1-qubit gates \eqref{Rn}.}
 \label{fig:GateCoords}
\end{figure}


\subsection{Pauli-power gates}\index{Pauli-power gates}
It turns out to be useful to define powers of the Pauli-gates. This is slightly tricky because non-integer powers of matrixes aren't unique. Just as there are 2-square roots of any number, a diagonalizable matrix with $n$ unique eigenvalue has $2^n$ unique square roots. We circumvent this ambiguity by defining the Pauli power gates via the Pauli rotation gates. We note that a $\pi$ rotation is a Pauli gate up to phase, e.g.
\[
R_{X}(\pi) = e^{-i \tfrac{\pi}{2}X } = -i X 
\]
and define powers of the Pauli matrices as 
\[
X^t =e^{-i \tfrac{\pi}{2} t (X-I)} \simeq R_x(\pi t) \ ,
\]
and similarly for $Y$ and $Z$ rotations. With this definition the Pauli-power gates spin states in the same direction around the Bloch sphere as the Pauli-rotation gates.

The Pauli rotation-representation is more natural from the point of view of pure mathematics. But the Pauli-power representation has computational advantages. % TODO: Rephrase
 In quantum circuits we most often encounter rotations of angles $\pm\pi /2^n$ for some integer $n$. Whereas it is easy to spot that $Z^{0.125}$ is a $T$ gate, for example, it is less obvious that $R_z(0.78538\ldots)$ is the same gate up to phase. Moreover binary fractions have exact floating point representations, whereas fractions of $\pi$ inevitably suffer from numerical round-off error.


\paragraph{X power gate}\index{X power gate}

\[
X^t & =  e^{-i \tfrac{\pi}{2} t (X-I)} 
        			= e^{i \tfrac{\pi}{2} t} R_x(\pi t)		
\\ \notag
			 &= e^{i \tfrac{\pi}{2} t} \begin{bmatrix*}[r]
                            \cos(\tfrac{\pi}{2}t) & -i \sin(\tfrac{\pi}{2}t) \\
                            -i \sin(\tfrac{\pi}{2}t) & \cos(\tfrac{\pi}{2}t)
                        \end{bmatrix*}
\]
\begin{center} \adjustbox{scale=0.8}{\begin{quantikz}[thin lines, column sep=0.75em,row sep={2.5em,between origins}]
& \gate{X^{t}} & \qw
\end{quantikz}
} \end{center}



\paragraph{Y power gate}\index{Y power gate}

\[
Y^t & =  e^{-i \tfrac{\pi}{2} t (Y-I)} 
        			= e^{i \tfrac{\pi}{2} t} R_y(\pi t)		
\\ \notag
	&= e^{i \tfrac{\pi}{2} t}
		\begin{bmatrix*}[r]
 			\cos(\tfrac{\pi}{2}t) & -\sin(\tfrac{\pi}{2}t)
        	\\ \sin(\tfrac{\pi}{2}t) & \cos(\tfrac{\pi}{2}t)
        \end{bmatrix*}
\]
\begin{center} \adjustbox{scale=0.8}{\begin{quantikz}[thin lines, column sep=0.75em,row sep={2.5em,between origins}]
& \gate{Y^{t}} & \qw
\end{quantikz}
} \end{center}




\paragraph{Z power gate}\index{Z power gate}

\[
Z^t & =  e^{-i \tfrac{\pi}{2} t (Z-I)} 
        			= e^{i \tfrac{\pi}{2} t} R_z(\pi t)		
\\ \notag
			 &= e^{i \tfrac{\pi}{2} t}         \begin{bmatrix*}
        e^{-i\tfrac{\pi}{2}t} & 0 \\
        0 & e^{+i\tfrac{\pi}{2}t}
                        \end{bmatrix*}
%
			 =       \begin{bmatrix*}
        1 & 0 \\
        0 & e^{+i \pi t}
                        \end{bmatrix*}                        
\]
\begin{center} \adjustbox{scale=0.8}{\begin{quantikz}[thin lines, column sep=0.75em,row sep={2.5em,between origins}]
& \gate{Z^{t}} & \qw
\end{quantikz}
} \end{center}



\paragraph{Phase shift gate}\cite{???}\index{phase shift gate} 
The name arrises because this gate shifts the phase of the $\ket{1}$ state relative to the $\ket{0}$ state.
\[
\label{phaseshift}
   P(\theta)  &=
           \begin{bmatrix*}
        1 & 0 \\
        0 & e^{i\theta}
                        \end{bmatrix*}
\\ & = e^{-i \half \theta}\ R_z(\theta)
\notag
\\ \notag
 & = Z^{\sfrac{\theta}{\pi}}
\]
%$$
%\adjustbox{scale=0.8}{\begin{quantikz}[thin lines, column sep=0.75em,row sep={2.5em,between origins}]
& \gate{R_z(\theta)} & \qw
\end{quantikz}
} FIXME
%$$

Sometimes favored over the $R_z$ gate because special values are exactly equal to various other common gates. For instance, $R_\pi =Z$, but $R_z(\pi) = -i Z$. The \Gate{CPhase} gate \eqref{CPhase} is a controlled phase shift. The phase shift gate also appears when considering the construction of controlled unitary gates~\secref{sec:ABCdeke}.

This gate is also commonly notated as $R_\theta$, but I have adopted the notation $\Gate{P}(\theta)$ (which is also used in qiskit and QASM~\cite{???}), in an attempt to reduce confusion with all the other ``R-subscript'' gates. Note that historically `P' was also used for the \Gate{S} gate, e.g.~\cite{???} \index{P gate|see {phase shift gate, S gate}}

\paragraph{Fractional phase shift gate}\cite{???}\index{fractional phase shift gate} 
Discrete fractional powers of the \Gate{Z} gate have their own notation. They most notably appear as controlled operations in the quantum Fourier transform~\secref{sec:QFT}.
\[
\label{fractionalphaseshift}
   P_k  &=
           \begin{bmatrix*}
        1 & 0 \\
        0 & e^{i 2\pi /2^k}
                        \end{bmatrix*}
\\ \notag
 &  = P(2 \pi / 2^k) = Z^{2^{1-k}}
 \\ 
 \notag P_1 &= Z \\ 
 \notag P_2 &= S \\
 \notag P_3 &= T 
\]
Thus $P_1$ is a half turn in the Bloch sphere, $P_2$ a quarter turn, $P_3$ an eighth turn, and so on. Most often notated as $R_k$, or sometimes as $Z_k$, here, as with the phase shift gate, I've adopted $P_k$ is a vain hope of reducing ambiguity.



\subsection{Quarter turns}
\todo{blurb}
\todo{express qV and h as V=HSH? h = ZH? ect...}

\paragraph{V gate}~\cite{???,???}\index{V gate}\index{square-root NOT|see{V gate}} Square root of the $X$-gate, $VV=X$. 
\[
V 
  & = X^{\half}
  \\ \notag
& = \half \begin{bmatrix*}[r] 1+i & 1-i \\ 1-i & 1+i \end{bmatrix*}
\\ \notag
& = H S H
\\ \notag
& \simeq R_x(+\tfrac{\pi}{2})
\label{V}
\]
\begin{center}
\adjustbox{scale=0.8}{\begin{quantikz}[thin lines, column sep=0.75em,row sep={2.5em,between origins}]
& \gate{V} & \qw
\end{quantikz}
}
 or 
\adjustbox{scale=0.8}{\begin{quantikz}[thin lines, column sep=0.75em,row sep={2.5em,between origins}]
& \gate{X^{\frac{1}{2}}} & \qw
\end{quantikz}
} 
\end{center}

\todo{Native gate}
\todo{Circuit}

A quarter turn anti-clockwise about the $\widehat{x}$ axis.
\begin{center}
\adjustbox{scale=0.75}{
 \begin{tikzpicture}[scale=1.5]
   \begin{scope}[canvas is zy plane at x=0]
     \draw (0,0) circle (1cm);
     %\draw[ultra thin] (-1,0) -- (1,0) (0,-1) -- (0,1);
     \draw[->] (0,0) -- (1.35,0) node[below ] {$\widehat{x}$};
   \end{scope}

   \begin{scope}[canvas is zx plane at y=0]
     \draw (0,0) circle (1cm);
     %\draw (-1,0) -- (1,0) (0,-1) -- (0,1);
     \draw[->] (0,0) -- (0,1.2) node[right] {$\widehat{y}$};
   \end{scope}

   \begin{scope}[canvas is xy plane at z=0]
     \draw (0,0) circle (1cm);
	%\draw (-1,0) -- (1,0) (0,-1) -- (0,1);
	\draw[->] (0,0) -- (0,1.2) node[above] {$\widehat{z}$};
   \end{scope}

 
    
   \draw[->] (1.5,0) -- node[above] {V} ++(1.0, 0) ;
 \end{tikzpicture}
 \begin{tikzpicture}[scale=1.5]
    \begin{scope}[canvas is zy plane at x=0]
     \draw (0,0) circle (1cm);
     %\draw[ultra thin] (-1,0) -- (1,0) (0,-1) -- (0,1);
     \draw[->] (0,0) -- (1.35,0) node[below ] {$\widehat{x}$};
   \end{scope}

   \begin{scope}[canvas is zx plane at y=0]
     \draw (0,0) circle (1cm);
     %\draw (-1,0) -- (1,0) (0,-1) -- (0,1);
     \draw[->] (0,0) -- (0,-1.2) node[left] {$\widehat{z}$};
   \end{scope}

   \begin{scope}[canvas is xy plane at z=0]
     \draw (0,0) circle (1cm);
	%\draw (-1,0) -- (1,0) (0,-1) -- (0,1);
	\draw[->] (0,0) -- (0,1.2) node[above] {$\widehat{y}$};
   \end{scope}
	 
 \end{tikzpicture}
 }
 \end{center}


\paragraph{Inverse V gate}\index{inverse V gate} Since the V-gate isn't Hermitian, the inverse gate, $V^\dagger$, is a distinct square root of $X$. 
\[
V^\dagger 
  & = X^{-\half}
\\ \notag
& = \half \begin{bmatrix*}[r] 1-i & 1+i \\ 1+i & 1-i \end{bmatrix*}
\\ \notag
& = H S^\dagger H
\\ \notag
& \simeq R_x(-\tfrac{\pi}{2})
\] 
\begin{center}
\adjustbox{scale=0.8}{\begin{quantikz}[thin lines, column sep=0.75em,row sep={2.5em,between origins}]
& \gate{V^\dagger} & \qw
\end{quantikz}
} or 
\adjustbox{scale=0.8}{\begin{quantikz}[thin lines, column sep=0.75em,row sep={2.5em,between origins}]
& \gate{X^{- \frac{1}{2}}} & \qw
\end{quantikz}
} 
\end{center}

\todo{Circuit}

A quarter turn clockwise about the $\widehat{x}$ axis.
\begin{center}
\adjustbox{scale=0.75}{
 \begin{tikzpicture}[scale=1.5]
   \begin{scope}[canvas is zy plane at x=0]
     \draw (0,0) circle (1cm);
     %\draw[ultra thin] (-1,0) -- (1,0) (0,-1) -- (0,1);
     \draw[->] (0,0) -- (1.35,0) node[below ] {$\widehat{x}$};
   \end{scope}

   \begin{scope}[canvas is zx plane at y=0]
     \draw (0,0) circle (1cm);
     %\draw (-1,0) -- (1,0) (0,-1) -- (0,1);
     \draw[->] (0,0) -- (0,1.2) node[right] {$\widehat{y}$};
   \end{scope}

   \begin{scope}[canvas is xy plane at z=0]
     \draw (0,0) circle (1cm);
	%\draw (-1,0) -- (1,0) (0,-1) -- (0,1);
	\draw[->] (0,0) -- (0,1.2) node[above] {$\widehat{z}$};
   \end{scope}

 

   \draw[->] (1.5,0) -- node[above] {$V^\dagger$} ++(1.0, 0) ;

 \begin{scope}[xshift=4.0cm] 
    \begin{scope}[canvas is zy plane at x=0]
     \draw (0,0) circle (1cm);
     %\draw[ultra thin] (-1,0) -- (1,0) (0,-1) -- (0,1);
     \draw[->] (0,0) -- (1.35,0) node[below ] {$\widehat{x}$};
   \end{scope}

   \begin{scope}[canvas is zx plane at y=0]
     \draw (0,0) circle (1cm);
     %\draw (-1,0) -- (1,0) (0,-1) -- (0,1);
     \draw[->] (0,0) -- (0,1.2) node[right] {$\widehat{z}$};
   \end{scope}

   \begin{scope}[canvas is xy plane at z=0]
     \draw (0,0) circle (1cm);
	%\draw (-1,0) -- (1,0) (0,-1) -- (0,1);
	\draw[->] (0,0) -- (0,-1.2) node[below] {$\widehat{y}$};
   \end{scope}
	 \end{scope}
 \end{tikzpicture}
 }
 \end{center}



\paragraph{Pseudo-Hadamard gate}~\cite{Jones1998a, Dorai2000a}\index{pseudo-Hadamard gate}:
Inverse square root of the Y-gate.\index{square-root Y-gate}
% https://arxiv.org/pdf/quant-ph/9805070.pdf # Descriptions
% https://arxiv.org/pdf/quant-ph/0006103.pdf # uses
\[
h & = \tfrac{\sqrt{2}}{1+i} Y^{-\half}
\\ \notag
& = \tfrac{1}{\sqrt{2}}\begin{bmatrix*}[r]1 & 1 \\ -1 & 1 \end{bmatrix*}
\]
\begin{center}
\adjustbox{scale=0.9}{\begin{quantikz}[thin lines, column sep=0.75em, row sep={2.5em,between origins}]
  & \gate{\text{h}} & \qw
  \end{quantikz}} 
or \adjustbox{scale=0.8}{\begin{quantikz}[thin lines, column sep=0.75em,row sep={2.5em,between origins}]
& \gate{Y^{{-\frac{{1}}{{2}}}}} & \qw
\end{quantikz}
}
\end{center}
A quarter turn clockwise about the $\widehat{y}$ axis.
\begin{center}
\adjustbox{scale=0.75}{
 \begin{tikzpicture}[scale=1.5]
   \begin{scope}[canvas is zy plane at x=0]
     \draw (0,0) circle (1cm);
     %\draw[ultra thin] (-1,0) -- (1,0) (0,-1) -- (0,1);
     \draw[->] (0,0) -- (1.35,0) node[below ] {$\widehat{x}$};
   \end{scope}

   \begin{scope}[canvas is zx plane at y=0]
     \draw (0,0) circle (1cm);
     %\draw (-1,0) -- (1,0) (0,-1) -- (0,1);
     \draw[->] (0,0) -- (0,1.2) node[right] {$\widehat{y}$};
   \end{scope}

   \begin{scope}[canvas is xy plane at z=0]
     \draw (0,0) circle (1cm);
	%\draw (-1,0) -- (1,0) (0,-1) -- (0,1);
	\draw[->] (0,0) -- (0,1.2) node[above] {$\widehat{z}$};
   \end{scope}


   \draw[->] (1.5,0) -- node[above] {$Y^{-\half}$} ++(1.0, 0) ;

 \begin{scope}[xshift=4.0cm] 
    \begin{scope}[canvas is zy plane at x=0]
     \draw (0,0) circle (1cm);
     %\draw[ultra thin] (-1,0) -- (1,0) (0,-1) -- (0,1);
     \draw[->] (0,0) -- (-1.35,0) node[right ] {$\widehat{z}$};
   \end{scope}

   \begin{scope}[canvas is zx plane at y=0]
     \draw (0,0) circle (1cm);
     %\draw (-1,0) -- (1,0) (0,-1) -- (0,1);
     \draw[->] (0,0) -- (0,1.2) node[right] {$\widehat{y}$};
   \end{scope}

   \begin{scope}[canvas is xy plane at z=0]
     \draw (0,0) circle (1cm);
	%\draw (-1,0) -- (1,0) (0,-1) -- (0,1);
	\draw[->] (0,0) -- (0,1.2) node[above] {$\widehat{x}$};
   \end{scope}
	 \end{scope}
 \end{tikzpicture}
 }
 \end{center}

This square-root of the $Y$-gate is called the pseudo-Hadamard gate as it has the same effect on the computational basis as the Hadamard gate. % TODO: RefSee p???.
\[
h\ket{0} & = \ket{+} \notag \\ 
h\ket{1} & = \ket{-} \notag
\]


\paragraph{Inverse pseudo-Hadamard gate}\index{inverse pseudo-Hadamard gate}\index{square-root Y-gate} Principle square root of the Y-gate. Unlike the Hadamard gate, the pseudo-Hadamard gate is not Hermitian, and therefore not its own inverse. 
\[
h^\dagger &=\frac{1}{\sqrt{2}}\begin{bmatrix*}[r] 1 & -1 \\ 1 & 1 \end{bmatrix*}
\\
& = \tfrac{\sqrt{2}}{1+i} Y^{\half} \notag
\]
\begin{center}
\adjustbox{scale=0.9}{\begin{quantikz}[thin lines, column sep=0.75em, row sep={2.5em,between origins}]
& \gate{\text{h$^\dagger$}} & \qw
\end{quantikz}
} or \adjustbox{scale=0.8}{\begin{quantikz}[thin lines, column sep=0.75em,row sep={2.5em,between origins}]
& \gate{Y^{{\frac{{1}}{{2}}}}} & \qw
\end{quantikz}
}
\end{center}

A quarter turn anti-clockwise about the $\widehat{y}$ axis.
\begin{center}
\adjustbox{scale=0.75}{
 \begin{tikzpicture}[scale=1.5]
   \begin{scope}[canvas is zy plane at x=0]
     \draw (0,0) circle (1cm);
     %\draw[ultra thin] (-1,0) -- (1,0) (0,-1) -- (0,1);
     \draw[->] (0,0) -- (1.35,0) node[below ] {$\widehat{x}$};
   \end{scope}

   \begin{scope}[canvas is zx plane at y=0]
     \draw (0,0) circle (1cm);
     %\draw (-1,0) -- (1,0) (0,-1) -- (0,1);
     \draw[->] (0,0) -- (0,1.2) node[right] {$\widehat{y}$};
   \end{scope}

   \begin{scope}[canvas is xy plane at z=0]
     \draw (0,0) circle (1cm);
	%\draw (-1,0) -- (1,0) (0,-1) -- (0,1);
	\draw[->] (0,0) -- (0,1.2) node[above] {$\widehat{z}$};
   \end{scope}


   \draw[->] (1.5,0) -- node[above] {$Y^{\half}$} ++(1.0, 0) ;

 \begin{scope}[xshift=4.0cm] 
    \begin{scope}[canvas is zy plane at x=0]
     \draw (0,0) circle (1cm);
     %\draw[ultra thin] (-1,0) -- (1,0) (0,-1) -- (0,1);
     \draw[->] (0,0) -- (1.35,0) node[below ] {$\widehat{z}$};
   \end{scope}

   \begin{scope}[canvas is zx plane at y=0]
     \draw (0,0) circle (1cm);
     %\draw (-1,0) -- (1,0) (0,-1) -- (0,1);
     \draw[->] (0,0) -- (0,1.2) node[right] {$\widehat{y}$};
   \end{scope}

   \begin{scope}[canvas is xy plane at z=0]
     \draw (0,0) circle (1cm);
	%\draw (-1,0) -- (1,0) (0,-1) -- (0,1);
	\draw[->] (0,0) -- (0,-1.2) node[below] {$\widehat{x}$};
   \end{scope}
	 \end{scope}
 \end{tikzpicture}
 }
 \end{center}



\paragraph{S gate} (Phase, P, ``ess'')
\index{S gate}%
\index{Phase gate|see {S gate}}%
\index{P gate|see {S gate}}%
\label{S}
 Square root of the $Z$-gate, $SS=Z$.
%\index{\adjustbox{scale=0.8}{\begin{quantikz}[thin lines, column sep=0.75em,row sep={2.5em,between origins}]
& \gate{S} & \qw
\end{quantikz}
}@S|see{S gate}}
\[ 
S & = Z^{\half} \\
&= \begin{bmatrix}1 & 0 \\ 0 & i \end{bmatrix}
\notag
\\ \notag
& \simeq R_z(+\tfrac{\pi}{2})
\]
\begin{center}
\adjustbox{scale=0.8}{\begin{quantikz}[thin lines, column sep=0.75em,row sep={2.5em,between origins}]
& \gate{S} & \qw
\end{quantikz}
}
\end{center}

Historically called the phase gate (and denoted by $P$), since it shifts the phase of the one state relative to the zero state. This is a bit confusing because we have to make the distinction between the phase gate and applying a global phase. Often referred to as simple the S (``ess'') gate in contemporary discourse. 


\begin{center}
\adjustbox{scale=0.75}{
 \begin{tikzpicture}[scale=1.5]
   \begin{scope}[canvas is zy plane at x=0]
     \draw (0,0) circle (1cm);
     %\draw[ultra thin] (-1,0) -- (1,0) (0,-1) -- (0,1);
     \draw[->] (0,0) -- (1.35,0) node[below ] {$\widehat{x}$};
   \end{scope}

   \begin{scope}[canvas is zx plane at y=0]
     \draw (0,0) circle (1cm);
     %\draw (-1,0) -- (1,0) (0,-1) -- (0,1);
     \draw[->] (0,0) -- (0,1.2) node[right] {$\widehat{y}$};
   \end{scope}

   \begin{scope}[canvas is xy plane at z=0]
     \draw (0,0) circle (1cm);
	%\draw (-1,0) -- (1,0) (0,-1) -- (0,1);
	\draw[->] (0,0) -- (0,1.2) node[above] {$\widehat{z}$};
   \end{scope}

 

   \draw[->] (1.5,0) -- node[above] {$S$} ++(1.0, 0) ;

 \begin{scope}[xshift=4.0cm] 
   \begin{scope}[canvas is zy plane at x=0]
     \draw (0,0) circle (1cm);
     %\draw[ultra thin] (-1,0) -- (1,0) (0,-1) -- (0,1);
     \draw[->] (0,0) -- (-1.35,0) node[right ] {$\widehat{y}$};
   \end{scope}

   \begin{scope}[canvas is zx plane at y=0]
     \draw (0,0) circle (1cm);
     %\draw (-1,0) -- (1,0) (0,-1) -- (0,1);
     \draw[->] (0,0) -- (0,1.2) node[right] {$\widehat{x}$};
   \end{scope}

   \begin{scope}[canvas is xy plane at z=0]
     \draw (0,0) circle (1cm);
	%\draw (-1,0) -- (1,0) (0,-1) -- (0,1);
	\draw[->] (0,0) -- (0,1.2) node[above] {$\widehat{z}$};
   \end{scope}
	 \end{scope}
 \end{tikzpicture}
 }
 \end{center}



\paragraph{Inverse S gate}\index{inverse S gate} Hermitian conjugate of the $S$ gate, and an alternative square-root of $Z$, $S^\dagger S^\dagger = Z$.
\[
S^\dagger 
& = Z^{-\half}
\\
&= \begin{bmatrix} 1 & 0 \\ 0 & -i \end{bmatrix}
\notag 
\\ \notag
& \simeq R_z(-\tfrac{\pi}{2})
\]
\begin{center}
\adjustbox{scale=0.8}{\begin{quantikz}[thin lines, column sep=0.75em,row sep={2.5em,between origins}]
& \gate{S^\dagger} & \qw
\end{quantikz}
}
\end{center}
\todo{alt Circuit}

A quarter turn clockwise about the $\widehat{z}$ axis.
\begin{center}
\adjustbox{scale=0.75}{
 \begin{tikzpicture}[scale=1.5]
   \begin{scope}[canvas is zy plane at x=0]
     \draw (0,0) circle (1cm);
     %\draw[ultra thin] (-1,0) -- (1,0) (0,-1) -- (0,1);
     \draw[->] (0,0) -- (1.35,0) node[below ] {$\widehat{x}$};
   \end{scope}

   \begin{scope}[canvas is zx plane at y=0]
     \draw (0,0) circle (1cm);
     %\draw (-1,0) -- (1,0) (0,-1) -- (0,1);
     \draw[->] (0,0) -- (0,1.2) node[right] {$\widehat{y}$};
   \end{scope}

   \begin{scope}[canvas is xy plane at z=0]
     \draw (0,0) circle (1cm);
	%\draw (-1,0) -- (1,0) (0,-1) -- (0,1);
	\draw[->] (0,0) -- (0,1.2) node[above] {$\widehat{z}$};
   \end{scope}

 

   \draw[->] (1.5,0) -- node[above] {$S^\dagger$} ++(1.0, 0) ;

 \begin{scope}[xshift=4.0cm] 
   \begin{scope}[canvas is zy plane at x=0]
     \draw (0,0) circle (1cm);
     %\draw[ultra thin] (-1,0) -- (1,0) (0,-1) -- (0,1);
     \draw[->] (0,0) -- (1.35,0) node[below ] {$\widehat{y}$};
   \end{scope}

   \begin{scope}[canvas is zx plane at y=0]
     \draw (0,0) circle (1cm);
     %\draw (-1,0) -- (1,0) (0,-1) -- (0,1);
     \draw[->] (0,0) -- (0,-1.2) node[left] {$\widehat{x}$};
   \end{scope}

   \begin{scope}[canvas is xy plane at z=0]
     \draw (0,0) circle (1cm);
	%\draw (-1,0) -- (1,0) (0,-1) -- (0,1);
	\draw[->] (0,0) -- (0,1.2) node[above] {$\widehat{z}$};
   \end{scope}

	 \end{scope}
 \end{tikzpicture}
 }
 \end{center}
Can be generated from the $S$ gate, $SSS= S^\dagger$.




\subsection{Hadamard gates}
\label{sec:Hadamard}

\paragraph{Hadamard gate}\index{Hadamard gate}
\index{H gate@\Gate{H} gate|see {Hadamard gate}}
\index{H|see {Hamiltonian}}
The Hadamard gate is one of the most interesting and useful of the common gates. Its effect is a $\pi$ rotation (half turn) in the Bloch sphere about the axis
$\tfrac{1}{\sqrt{2}}(\widehat{x}+\widehat{z})$. % FIXME: \hat not working!?
In a sense the Hadamard gate is half way between the Z and X gates (Fig.~\ref{fig:GateCoords}).
%
\[
H & = \tfrac{1}{\sqrt{2}}\begin{bmatrix*}[r]1 & 1 \\ 1 & -1 \end{bmatrix*} \\
	& \simeq R_{\vec{n}}(\pi), \quad \vec{n} = \tfrac{1}{\sqrt{2}}(1, 0, 1) \notag
\]

\begin{center}
\adjustbox{scale=0.8}{\begin{quantikz}[thin lines, column sep=0.75em,row sep={2.5em,between origins}]
& \gate{H} & \qw
\end{quantikz}
}
\end{center}



In terms of the Bloch sphere, the Hadamard gate interchanges the $\widehat{x}$ and $\widehat{z}$ axes, and inverts the $\widehat{y}$ axis.
\makeatletter
\tikzoption{canvas is plane}[]{\@setOxy#1}
\def\@setOxy O(#1,#2,#3)x(#4,#5,#6)y(#7,#8,#9)%
  {\def\tikz@plane@origin{\pgfpointxyz{#1}{#2}{#3}}%
   \def\tikz@plane@x{\pgfpointxyz{#4}{#5}{#6}}%
   \def\tikz@plane@y{\pgfpointxyz{#7}{#8}{#9}}%
   \tikz@canvas@is@plane
  }
\makeatother  

\begin{center}
\adjustbox{scale=0.75}{
 \begin{tikzpicture}[scale=1.5]
   \begin{scope}[canvas is zy plane at x=0]
     \draw (0,0) circle (1cm);
     %\draw[ultra thin] (-1,0) -- (1,0) (0,-1) -- (0,1);
     \draw[->] (0,0) -- (1.35,0) node[below ] {$\widehat{x}$};
     \draw[dashed] (0,0) -- (0.707,0.707) node {$\bullet$} ;
   \end{scope}

   \begin{scope}[canvas is zx plane at y=0]
     \draw (0,0) circle (1cm);
     %\draw (-1,0) -- (1,0) (0,-1) -- (0,1);
     \draw[->] (0,0) -- (0,1.2) node[right] {$\widehat{y}$};
   \end{scope}

   \begin{scope}[canvas is xy plane at z=0]
     \draw (0,0) circle (1cm);
	%\draw (-1,0) -- (1,0) (0,-1) -- (0,1);
	\draw[->] (0,0) -- (0,1.2) node[above] {$\widehat{z}$};
   \end{scope}

   \begin{scope}[canvas is plane={O(0,0.707,0.707)x(1,0,0)y(0,1,-1)}]
   	\centerarc[blue,->](0,0)(105:285:0.25)
   \end{scope}  
    
   \draw[->] (1.5,0) -- node[above] {H} ++(1.0, 0) ;
 \end{tikzpicture}
 \begin{tikzpicture}[scale=1.5]
   \begin{scope}[canvas is zy plane at x=0]
     \draw (0,0) circle (1cm);
     %\draw[ultra thin] (-1,0) -- (1,0) (0,-1) -- (0,1);
     \draw[->] (0,0) -- (1.35,0) node[below ] {$\widehat{z}$};
   \end{scope}

   \begin{scope}[canvas is zx plane at y=0]
     \draw (0,0) circle (1cm);
     %\draw (-1,0) -- (1,0) (0,-1) -- (0,1);
     \draw[->] (0,0) -- (0,-1.2) node[left] {$\widehat{y}$};
   \end{scope}

   \begin{scope}[canvas is xy plane at z=0]
     \draw (0,0) circle (1cm);
	%\draw (-1,0) -- (1,0) (0,-1) -- (0,1);
	\draw[->] (0,0) -- (0,1.2) node[above] {$\widehat{x}$};
   \end{scope}

	 
 \end{tikzpicture}
 }
 \end{center}
 
A Hadamard similarity transform interchanges $X$ and $Z$ gates,
\[
HXH = Z, \qquad HYH &= -Y, \qquad HZH = X \notag 
\\
%\]
%\[
H R_x(\theta) H = R_z(\theta) , \qquad H R_y(\theta) H &= R_y(-\theta) , \qquad H R_z(\theta) H = R_x(\theta) 
\notag
\]

\todo{Rotation gate effect follows form Taylor expansion}


One reason that the Hadamard gate is so useful is that it acts on the computation basis states to create superpositions of zero and one states. These states are common enough that they have their own notation, $\ket{+}$ and \ket{-}.
\[
H\ket{0} & = \tfrac{1}{\sqrt{2}}(\ket{0}+\ket{1}) = \ket{+}
\notag
\\
H\ket{1} & = \tfrac{1}{\sqrt{2}}(\ket{0}-\ket{1}) = \ket{-} 
\notag
\]
The square of the Hadamard gate is the identity $HH=I$. This is easy to show with some simple algebra, or by considering that the Hadamard is a 180 degree rotation in the Bloch sphere, or by noting that the Hadamard matrix is both Hermitian and unitary, so the Hadamard must be its own inverse. As a consequence, the Hadamard converts the \ket{+}, \ket{-} Hadamard basis back to the \ket{0}, \ket{1} computational basis.
\index{computational basis} \index{Hadamard basis} \index{$\ket{+}$}\index{$\ket{-}$}
\[
H\ket{+} & = \ket{0} \notag \\ 
H\ket{-} & = \ket{1} \notag
\notag
\]

The Hadamard gate is named for the \define{Hadamard transform} (Or \define{Walsh-Hadamard transform}), which in the context of quantum computing is the simultaneous application of Hadamard gates to multiple-qubits. We will return this transform presently \secref{???}. The Hadamard gate is also the 1-qubit quantum Fourier transform \secref{???}\index{Hadamard transform}\index{Walsh-Hadamard transform|see {Hadamard transform}} \index{quantum Fourier transform! 1-qubit}
% TODO: Is this correct? Of is general Walsh-Hadamard QFT?

It is also worth noting a couple of useful decompositions (up to phase).
\begin{center}
\adjustbox{scale=0.8}{\begin{quantikz}[thin lines, column sep=0.75em,row sep={2.5em,between origins}]
& \gate{H} & \qw
\end{quantikz}
} $\simeq$
\adjustbox{scale=0.8}{\begin{quantikz}[thin lines, column sep=0.75em,row sep={2.5em,between origins}]
& \gate{Z} & \gate{Y^{\frac{1}{2}}} & \qw
\end{quantikz}
}
\end{center}


\begin{center}
\adjustbox{scale=0.75}{
 \begin{tikzpicture}[scale=1.5]
   \begin{scope}[canvas is zy plane at x=0]
     \draw (0,0) circle (1cm);
     %\draw[ultra thin] (-1,0) -- (1,0) (0,-1) -- (0,1);
     \draw[->] (0,0) -- (1.35,0) node[below] {$\widehat{x}$};
%     \draw[dashed] (0,0) -- (1.,1.) ;
   \end{scope}

   \begin{scope}[canvas is zx plane at y=0]
     \draw (0,0) circle (1cm);
     %\draw (-1,0) -- (1,0) (0,-1) -- (0,1);
     \draw[->] (0,0) -- (0,1.175) node[right] {$\widehat{y}$};
   \end{scope}

   \begin{scope}[canvas is xy plane at z=0]
     \draw (0,0) circle (1cm);
	%\draw (-1,0) -- (1,0) (0,-1) -- (0,1);
	\draw[->] (0,0) -- (0,1.175) node[above] {$\widehat{z}$};
   \end{scope}

%   \begin{scope}[canvas is zx plane at y=1.0]	
%   	\centerarc[blue,->](0,0)(270:90:0.25)
%   \end{scope}
    
   \draw[->] (1.5,0) -- node[above] {$Z$} ++(0.5, 0) ;

	\begin{scope}[xshift=3.5cm]
   \begin{scope}[canvas is zy plane at x=0]
     \draw (0,0) circle (1cm);
     %\draw[ultra thin] (-1,0) -- (1,0) (0,-1) -- (0,1);
     \draw[->] (0,0) -- (-1.35,0) node[above] {$\widehat{x}$};
   \end{scope}

   \begin{scope}[canvas is zx plane at y=0]
     \draw (0,0) circle (1cm);
     %\draw (-1,0) -- (1,0) (0,-1) -- (0,1);
     \draw[->] (0,0) -- (0,-1.175) node[left] {$\widehat{y}$};
   \end{scope}

   \begin{scope}[canvas is xy plane at z=0]
     \draw (0,0) circle (1cm);
	%\draw (-1,0) -- (1,0) (0,-1) -- (0,1);
	\draw[->] (0,0) -- (0,1.175) node[above] {$\widehat{z}$};
   \end{scope}
   \draw[->] (1.5,0) -- node[above] {$Y^{\frac{1}{2}}$} ++(0.5, 0) ;

	\end{scope}


	\begin{scope}[xshift=7.0cm]
   \begin{scope}[canvas is zy plane at x=0]
     \draw (0,0) circle (1cm);
     %\draw[ultra thin] (-1,0) -- (1,0) (0,-1) -- (0,1);
     \draw[->] (0,0) -- (1.35,0) node[below ] {$\widehat{z}$};
   \end{scope}

   \begin{scope}[canvas is zx plane at y=0]
     \draw (0,0) circle (1cm);
     %\draw (-1,0) -- (1,0) (0,-1) -- (0,1);
     \draw[->] (0,0) -- (0,-1.175) node[left] {$\widehat{y}$};
   \end{scope}

   \begin{scope}[canvas is xy plane at z=0]
     \draw (0,0) circle (1cm);
	%\draw (-1,0) -- (1,0) (0,-1) -- (0,1);
	\draw[->] (0,0) -- (0,1.175) node[above] {$\widehat{x}$};
   \end{scope}

	\end{scope}
 \end{tikzpicture}
}
 \end{center}

\begin{center}
\adjustbox{scale=0.8}{\begin{quantikz}[thin lines, column sep=0.75em,row sep={2.5em,between origins}]
& \gate{H} & \qw
\end{quantikz}
} $\simeq$
\adjustbox{scale=0.8}{\begin{quantikz}[thin lines, column sep=0.75em,row sep={2.5em,between origins}]
& \gate{S} & \gate{V} & \gate{S} & \qw
\end{quantikz}
}
\end{center}


\begin{center}
\adjustbox{scale=0.55}{
 \begin{tikzpicture}[scale=1.5]
   \begin{scope}[canvas is zy plane at x=0]
     \draw (0,0) circle (1cm);
     %\draw[ultra thin] (-1,0) -- (1,0) (0,-1) -- (0,1);
     \draw[->] (0,0) -- (1.35,0) node[below] {$\widehat{x}$};
%     \draw[dashed] (0,0) -- (1.,1.) ;
   \end{scope}

   \begin{scope}[canvas is zx plane at y=0]
     \draw (0,0) circle (1cm);
     %\draw (-1,0) -- (1,0) (0,-1) -- (0,1);
     \draw[->] (0,0) -- (0,1.35) node[above] {$\widehat{y}$};
   \end{scope}

   \begin{scope}[canvas is xy plane at z=0]
     \draw (0,0) circle (1cm);
	%\draw (-1,0) -- (1,0) (0,-1) -- (0,1);
	\draw[->] (0,0) -- (0,1.175) node[above] {$\widehat{z}$};
   \end{scope}

%   \begin{scope}[canvas is zx plane at y=1.0]	
%   	\centerarc[blue,->](0,0)(270:90:0.25)
%   \end{scope}
    
   \draw[->] (1.5,0) -- node[above] {$S$} ++(0.5, 0) ;

	\begin{scope}[xshift=3.5cm]
    \begin{scope}[canvas is zy plane at x=0]
     \draw (0,0) circle (1cm);
     %\draw[ultra thin] (-1,0) -- (1,0) (0,-1) -- (0,1);
     \draw[->] (0,0) -- (-1.35,0) node[right] {$\widehat{y}$};
%     \draw[dashed] (0,0) -- (1.,1.) ;
   \end{scope}

   \begin{scope}[canvas is zx plane at y=0]
     \draw (0,0) circle (1cm);
     %\draw (-1,0) -- (1,0) (0,-1) -- (0,1);
     \draw[->] (0,0) -- (0,1.175) node[right] {$\widehat{x}$};
   \end{scope}

   \begin{scope}[canvas is xy plane at z=0]
     \draw (0,0) circle (1cm);
	%\draw (-1,0) -- (1,0) (0,-1) -- (0,1);
	\draw[->] (0,0) -- (0,1.175) node[above] {$\widehat{z}$};
   \end{scope}
      \draw[->] (1.5,0) -- node[above] {$V$} ++(0.5, 0) ;

	\end{scope}


	\begin{scope}[xshift=7.0cm]
   \begin{scope}[canvas is zy plane at x=0]
     \draw (0,0) circle (1cm);
     %\draw[ultra thin] (-1,0) -- (1,0) (0,-1) -- (0,1);
     \draw[->] (0,0) -- (-1.35,0) node[right] {$\widehat{y}$};
   \end{scope}

   \begin{scope}[canvas is zx plane at y=0]
     \draw (0,0) circle (1cm);
     %\draw (-1,0) -- (1,0) (0,-1) -- (0,1);
     \draw[->] (0,0) -- (0,-1.175) node[left] {$\widehat{z}$};
   \end{scope}

   \begin{scope}[canvas is xy plane at z=0]
     \draw (0,0) circle (1cm);
	%\draw (-1,0) -- (1,0) (0,-1) -- (0,1);
	\draw[->] (0,0) -- (0,1.175) node[above] {$\widehat{x}$};
   \end{scope}
   
   \draw[->] (1.5,0) -- node[above] {$S$} ++(0.5, 0) ;

	\end{scope}
	\begin{scope}[xshift=10.5cm]
   \begin{scope}[canvas is zy plane at x=0]
     \draw (0,0) circle (1cm);
     %\draw[ultra thin] (-1,0) -- (1,0) (0,-1) -- (0,1);
     \draw[->] (0,0) -- (1.35,0) node[below ] {$\widehat{z}$};
   \end{scope}

   \begin{scope}[canvas is zx plane at y=0]
     \draw (0,0) circle (1cm);
     %\draw (-1,0) -- (1,0) (0,-1) -- (0,1);
     \draw[->] (0,0) -- (0,-1.175) node[left] {$\widehat{y}$};
   \end{scope}

   \begin{scope}[canvas is xy plane at z=0]
     \draw (0,0) circle (1cm);
	%\draw (-1,0) -- (1,0) (0,-1) -- (0,1);
	\draw[->] (0,0) -- (0,1.175) node[above] {$\widehat{x}$};
   \end{scope}

	\end{scope}	
 \end{tikzpicture}
}
 \end{center}
 
Here, $V$ is the square-root of the $X$ gate, and $S$ is the square-root of Z, each of which is a quarter turn in the Bloch sphere.
 
 
% [TODO (elsewhere?): H is 1-qubit QFT. Multiqubit hadamard transform]   
% TODO
% 1-qubit Fourier transform
% Named for Hadamard transform (Walsh-Hadamard transform) 
% Explain +/- states, "polar basis" Hadamard basis? Also angled arrows?
% Create supposition of all computational basis states
% Hermitian, so own inverse
% When introduced into quantum computing?



\paragraph{Hadamard-like gates}
\index{Hadamard-like gates}
If we peruse the sphere of 1-qubit gates, Fig.~\ref{fig:GateCoords}, we can see that there are 6 different Hadamard-like gates that lie between the main \axis{x}, \axis{y}, and \axis{z} axes. (Recall that gates on opposite sides of the sphere's surface are the same up to phase.) Each of these gates can be obtained from straightforward transform so the Hadamard gate. For instance, $H_{YZ} = S H S^\dagger$ is the Hadamard-like gate between the $Z$ and $Y$ gates, which interchanges the \axis{y} and \axis{z} axies, and flips the \axis{x}-axis.


\begin{center}
\adjustbox{scale=0.75}{
 \begin{tikzpicture}[scale=1.5]
   \begin{scope}[canvas is zy plane at x=0]
     \draw (0,0) circle (1cm);
     %\draw[ultra thin] (-1,0) -- (1,0) (0,-1) -- (0,1);
     \draw[->] (0,0) -- (1.35,0) node[below ] {$\widehat{x}$};

   \end{scope}

   \begin{scope}[canvas is zx plane at y=0]
     \draw (0,0) circle (1cm);
     %\draw (-1,0) -- (1,0) (0,-1) -- (0,1);
     \draw[->] (0,0) -- (0,1.2) node[right] {$\widehat{y}$};
   \end{scope}

   \begin{scope}[canvas is xy plane at z=0]
     \draw (0,0) circle (1cm);
	%\draw (-1,0) -- (1,0) (0,-1) -- (0,1);
	\draw[->] (0,0) -- (0,1.2) node[above] {$\widehat{z}$};
	     \draw[dashed] (0,0) -- (0.707,0.707) node {$\bullet$};
   \end{scope}

%       \begin{scope}[canvas is plane={O(0.707,0.707,0)x(0,1,-1)y(1,0,0)}]
%       	\centerarc[blue,->](0,0)(290:105:0.25)
%       \end{scope}  

  \begin{scope}[canvas is plane={O(0.707,0.707,0)x(0,1,0)y(1,0,0)}]
   	\centerarc[thick, blue,->](0,0)(310:125:0.25)
	\draw (0,0) node {$\bullet$};
   \end{scope}  
             
    
   \draw[->] (1.5,0) -- node[above] {$S H S^\dagger$} ++(1.0, 0) ;
 \end{tikzpicture}
 \begin{tikzpicture}[scale=1.5]
   \begin{scope}[canvas is zy plane at x=0]
     \draw (0,0) circle (1cm);
     %\draw[ultra thin] (-1,0) -- (1,0) (0,-1) -- (0,1);
     \draw[->] (0,0) -- (-1.35,0) node[above ] {$\widehat{x}$};
   \end{scope}

   \begin{scope}[canvas is zx plane at y=0]
     \draw (0,0) circle (1cm);
     %\draw (-1,0) -- (1,0) (0,-1) -- (0,1);
     \draw[->] (0,0) -- (0,1.2) node[right] {$\widehat{z}$};
   \end{scope}

   \begin{scope}[canvas is xy plane at z=0]
     \draw (0,0) circle (1cm);
	%\draw (-1,0) -- (1,0) (0,-1) -- (0,1);
	\draw[->] (0,0) -- (0,1.2) node[above] {$\widehat{y}$};
   \end{scope}

	 
 \end{tikzpicture}
 }
 \end{center}


\begin{figure}[tp]
%\begin{figure}[t]
\begin{center}
 \begin{tikzpicture}[scale=3]
   \begin{scope}[canvas is zy plane at x=0]
     \draw (0,0) circle (1cm);
     \draw (-1,0) -- (1,0) (0,-1) -- (0,1);
     \draw[->] (0,0) -- (1.2,0) node[ below left] {$\theta\ n_x$};     
   \end{scope}

   \begin{scope}[canvas is zx plane at y=0]
     \draw (0,0) circle (1cm);
     \draw (-1,0) -- (1,0) (0,-1) -- (0,1);
     \draw[->] (0,0) -- (0,1.2) node[right ] {$\theta\ n_y$};     
   \end{scope}

   \begin{scope}[canvas is xy plane at z=0]
     \draw (0,0) circle (1cm);
     \draw (-1,0) -- (1,0) (0,-1) -- (0,1);
     \draw[->] (0,0) -- (0,1.2) node[above ] {$\theta\ n_z$};     
   \end{scope}

%	\node[fill=white] at (0,0,0) {$I$};
%
%	\node[fill=white] at (0,1,0) {$Z$};
%	\node[fill=white] at (0,0.5,0) {$S$};
%	\node[fill=white] at (0,0.25,0) {$T$};			
%	\node[fill=white] at (0,-0.5,0) {${S^\dagger}$};
%	\node[fill=white] at (0,-0.25,0) {$T^\dagger$};	
%	% \node[fill=white] at (0,-1,0) {Z};
%		
%	\node[fill=white] at (0,0,1) {$X$};
%	\node[fill=white] at (0,0,0.5) {$V$};
%	\node[fill=white] at (0,0,-0.5) {$V^\dagger$};	
%	% \node[fill=white] at (0,0,-1) {X};
%	
	\node[fill=white] at (0,{sqrt(1/2)},{sqrt(1/2)}) {$H$};
	\node[fill=white] at ({sqrt(1/2)},0, {sqrt(1/2)}) {$VHV^\dagger$};	
	\node[fill=white] at ({sqrt(1/2)},{sqrt(1/2)}, 0) {$SHS^\dagger$};	
	\node[fill=white] at (0,{-sqrt(1-0.64)}, {0.8}) {$Y^{\frac{1}{2}} H Y^{\text{-}\frac{1}{2}}$};
	\node[fill=white] at ({-sqrt(1-0.64)}, 0, {0.8}) {$V^\dagger H V$};	
	\node[fill=white] at ({-sqrt(1/2)},{sqrt(1/2)}, 0) {$S^\dagger H S$};
			
%%	\node[fill=white] at (0,0,1) {H};
%	
%	\node[fill=white] at (1,0,0) {$Y$};
%	\node[fill=white] at (0.5,0,0) {${h^\dagger}$};	
%	\node[fill=white] at (-0.5,0,0) {${h}$};	
	% \node[fill=white] at (-1,0,0) {$Y$};
 \end{tikzpicture}
 \end{center}
 \label{fig:GateCoordsHadamard}
\caption{Coordinates of the 6-Hadamard like gates. \todo{CHECK!}}
\end{figure}
This particular Hadamard-like gate takes the computational Z-basis to the Y-basis.
\[
S H S^\dagger \ket{0} =  \tfrac{1}{\sqrt{2}}(\ket{0}+i\ket{1})  = \ket{+i} \notag \\
S H S^\dagger \ket{1} =  \tfrac{1}{\sqrt{2}}(\ket{0}-i\ket{1})  = \ket{-i} \notag
\]
% These Hadamard-like gates are all Clifford gates (\S \ref{ChClifford}), and 
The coordinates of all 6 Hadamard-like gates are shown in Fig.~\ref{fig:GateCoordsHadamard}, and listed in Table~\ref{tab:Clifford1q} in the same block as the Hadamard gate. 

\todo{Maybe move to Clifford chapter}

%%
%%\paragraph{Hadamard power gate}
%%\todo{Define via Rn}
%%\todo{Not needed?}
%%
%%\[
%%H_H = \frac{\pi}{2} (\frac{1}{\sqrt{2}}(X + Z) - I)
%%\notag
%%\]
%%
%%\[
%% H^t = e^{i \pi t/2}
%%        \begin{bmatrix*}[r]
%%            \cos(\tfrac{t}{2}) + \tfrac{i}{\sqrt{2}}\sin(\tfrac{t}{2}) &
%%            \tfrac{i}{\sqrt{2}} \sin(\tfrac{t}{2}) \\
%%            \tfrac{i}{\sqrt{2}} \sin(\tfrac{t}{2}) &
%%            \cos(\tfrac{t}{2}) -\tfrac{i}{\sqrt{2}} \sin(\frac{t}{2})
%%        \end{bmatrix*}
%%\]
%%
%%
%%\begin{center}
%%\adjustbox{scale=0.75}{
%% \begin{tikzpicture}[scale=1.5]
%%   \begin{scope}[canvas is zy plane at x=0]
%%     \draw (0,0) circle (1cm);
%%     %\draw[ultra thin] (-1,0) -- (1,0) (0,-1) -- (0,1);
%%     \draw[->] (0,0) -- (1.35,0) node[below ] {$\widehat{x}$};
%%     \draw[dashed] (0,0) -- (0.707,0.707) node {$\bullet$} ;
%%   \end{scope}
%%
%%   \begin{scope}[canvas is zx plane at y=0]
%%     \draw (0,0) circle (1cm);
%%     %\draw (-1,0) -- (1,0) (0,-1) -- (0,1);
%%     \draw[->] (0,0) -- (0,1.2) node[right] {$\widehat{y}$};
%%   \end{scope}
%%
%%   \begin{scope}[canvas is xy plane at z=0]
%%     \draw (0,0) circle (1cm);
%%	%\draw (-1,0) -- (1,0) (0,-1) -- (0,1);
%%	\draw[->] (0,0) -- (0,1.2) node[above] {$\widehat{z}$};
%%   \end{scope}
%%
%%   \begin{scope}[canvas is plane={O(0,0.707,0.707)x(1,0,0)y(0,1,-1)}]
%%   	\centerarc[blue,->](0,0)(105:285:0.25)
%%   \end{scope}  
%%    
%%	 
%% \end{tikzpicture}
%% }
%% \end{center}
%%%

\subsection{Axis cycling gates}
Another interesting, but rarely discussed\footnote{Period 3 axis cycling gates are widely discussed abstractly in the context of Clifford gates \secref{sec:Clifford}. I've borrowed the explicit realization and nomenclature from 
Craig Gidney's \texttt{stim} python package, a simulator for quantum stabilizer circuits. \url{https://github.com/quantumlib/Stim} \cite{Gidney2021a}} class of gates are those that interchange three axes. These gates have periodicity 3 and represent 120 degree rotations of the Bloch sphere.
\index{axis cycling gates}\index{C gate|see{axis cycling gates}}

\paragraph{C gate}~\cite{Gidney2021a}
\[
C & = \tfrac{1}{2} \begin{bmatrix}+1-i & -1-i \\ +1-i &+1+i \end{bmatrix}
\\ \notag 
& = R_n(\tfrac{2}{3}\pi),\quad  n = (\tfrac{1}{\sqrt{3}},\tfrac{1}{\sqrt{3}},\tfrac{1}{\sqrt{3}})
\]
A right handed period 3 axis cycling gate, cycling the axes in the permutation $\axis{x}\rightarrow \axis{y} \rightarrow \axis{z} \rightarrow \axis{x}$
\[
C\ X^t\ C^\dagger &= Y^t \notag \\
C\ Y^t\ C^\dagger &= Z^t \notag \\
C\ Z^t\ C^\dagger &= X^t \notag
\]
Note that this is a third root of the identity, 
$C^3 = I$, and that the square gives the inverse gate $C^2 = C^\dagger$ which cycles in the opposite direction. 

There are 8 distinct axis cycling gates, which are all also Clifford gates and listed in the last block of table~\ref{tab:Clifford1q}. Each such gate can be broken down into a combination of two quarter turns, e.g.~$C=SV$.


\begin{center}
\adjustbox{scale=0.75}{
 \begin{tikzpicture}[scale=1.5]
   \begin{scope}[canvas is zy plane at x=0]
     \draw (0,0) circle (1cm);
     %\draw[ultra thin] (-1,0) -- (1,0) (0,-1) -- (0,1);
     \draw[->] (0,0) -- (1.35,0) node[below ] {$\widehat{x}$};
   \end{scope}

   \begin{scope}[canvas is zx plane at y=0]
     \draw (0,0) circle (1cm);
     %\draw (-1,0) -- (1,0) (0,-1) -- (0,1);
     \draw[->] (0,0) -- (0,1.2) node[right] {$\widehat{y}$};
   \end{scope}

   \begin{scope}[canvas is xy plane at z=0]
     \draw (0,0) circle (1cm);
	%\draw (-1,0) -- (1,0) (0,-1) -- (0,1);
	\draw[->] (0,0) -- (0,1.2) node[above] {$\widehat{z}$};
   \end{scope}

 
    
   \draw[->] (1.5,0) -- node[above] {V} ++(0.75, 0) ;
 \end{tikzpicture}
 \begin{tikzpicture}[scale=1.5]
    \begin{scope}[canvas is zy plane at x=0]
     \draw (0,0) circle (1cm);
     %\draw[ultra thin] (-1,0) -- (1,0) (0,-1) -- (0,1);
     \draw[->] (0,0) -- (1.35,0) node[below ] {$\widehat{x}$};
   \end{scope}

   \begin{scope}[canvas is zx plane at y=0]
     \draw (0,0) circle (1cm);
     %\draw (-1,0) -- (1,0) (0,-1) -- (0,1);
     \draw[->] (0,0) -- (0,-1.2) node[left] {$\widehat{z}$};
   \end{scope}

   \begin{scope}[canvas is xy plane at z=0]
     \draw (0,0) circle (1cm);
	%\draw (-1,0) -- (1,0) (0,-1) -- (0,1);
	\draw[->] (0,0) -- (0,1.2) node[above] {$\widehat{y}$};
   \end{scope}
	   \draw[->] (1.5,0) -- node[above] {S} ++(0.75, 0) ;
 
 \end{tikzpicture}
 
 \begin{tikzpicture}[scale=1.5]
   \begin{scope}[canvas is zy plane at x=0]
     \draw (0,0) circle (1cm);
     %\draw[ultra thin] (-1,0) -- (1,0) (0,-1) -- (0,1);
     \draw[->] (0,0) -- (1.35,0) node[below ] {$\widehat{z}$};
   \end{scope}

   \begin{scope}[canvas is zx plane at y=0]
     \draw (0,0) circle (1cm);
     %\draw (-1,0) -- (1,0) (0,-1) -- (0,1);
     \draw[->] (0,0) -- (0,1.2) node[right] {$\widehat{x}$};
   \end{scope}

   \begin{scope}[canvas is xy plane at z=0]
     \draw (0,0) circle (1cm);
	%\draw (-1,0) -- (1,0) (0,-1) -- (0,1);
	\draw[->] (0,0) -- (0,1.2) node[above] {$\widehat{y}$};
   \end{scope}
 
 \end{tikzpicture}
 }
 \end{center}



\subsection{T gates}
All the of preceding discrete 1-qubit gates (Pauli gates, quarter turns, Hadamard and Hadamard-like gates, and axis cycling gates) are examples of a special class of gates called Clifford gates. Although important, the Clifford gates have the notable restricting that they aren't universal -- you can't build an arbitrary qubit rotation from Clifford gates alone. The is because the Clifford gates always map the \axis{x}, \axis{y} and \axis{z} axes back onto themselves. In order to be computational universal, it is necessary to have at least one non-Clifford gate in our gate set, and the most common choice for that non-Clifford gate is the $T$ gate, one eighth of a rotation anti-clockwise about the $z$ axis. A gate set consisting of all Cliffords (including multi-qubit Cliffords) and the T gate is often written as ``Clifford+T''.
\index{Clifford+T}\index{Clifford gates}

\todo{Wordsmith}


\paragraph{T gate} ("tee", $\pi/8$) ~\cite{???,???} Forth root of the $Z$ gate, $T^4=Z$.
\index{T gate}
\index{$\pi/8$ gate | see {T gate}}

\[
T & = Z^{\frac{1}{4}} \\
\notag
& = \begin{bmatrix}1 & 0 \\ 0 & e^{i \frac{\pi}{4}} \end{bmatrix}
\]
\begin{center}
\adjustbox{scale=0.8}{\begin{quantikz}[thin lines, column sep=0.75em,row sep={2.5em,between origins}]
& \gate{T} & \qw
\end{quantikz}
}
\end{center}

The T gate has sometimes been called the $\pi/8$ gate since we can extract a phase and write the T gate as
\[
T = e^{i\tfrac{\pi}{8}\pi} \begin{bmatrix} e^{-i\tfrac{\pi}{8}} & 0 \\ 0 & e^{+i\tfrac{\pi}{8}} 
\end{bmatrix}
\notag
\]

An eight turn anti-clockwise about the $\widehat{z}$ axis.
\begin{center}
\adjustbox{scale=0.75}{
 \begin{tikzpicture}[scale=1.5]
   \begin{scope}[canvas is zy plane at x=0]
     \draw (0,0) circle (1cm);
     %\draw[ultra thin] (-1,0) -- (1,0) (0,-1) -- (0,1);
     \draw[->] (0,0) -- (1.35,0) node[below ] {$\widehat{x}$};
   \end{scope}

   \begin{scope}[canvas is zx plane at y=0]
     \draw (0,0) circle (1cm);
     %\draw (-1,0) -- (1,0) (0,-1) -- (0,1);
     \draw[->] (0,0) -- (0,1.2) node[right] {$\widehat{y}$};
   \end{scope}

   \begin{scope}[canvas is xy plane at z=0]
     \draw (0,0) circle (1cm);
	%\draw (-1,0) -- (1,0) (0,-1) -- (0,1);
	\draw[->] (0,0) -- (0,1.2) node[above] {$\widehat{z}$};
   \end{scope}

 
    
   \draw[->] (1.5,0) -- node[above] {$T^\dagger$} ++(1.0, 0) ;
 \end{tikzpicture}
 \begin{tikzpicture}[scale=1.5]
   \begin{scope}[canvas is zy plane at x=0]
     \draw (0,0) circle (1cm);

   \end{scope}

   \begin{scope}[canvas is zx plane at y=0]
     \draw (0,0) circle (1cm);
     %\draw (-1,0) -- (1,0) (0,-1) -- (0,1);
     \draw[->] (0,0) -- (0.807,0.807) node[right] {$\widehat{x}$};
     \draw[->] (0,0) -- (-0.807,0.807) node[right] {$\widehat{y}$};     
   \end{scope}

   \begin{scope}[canvas is xy plane at z=0]
     \draw (0,0) circle (1cm);
	%\draw (-1,0) -- (1,0) (0,-1) -- (0,1);
	\draw[->] (0,0) -- (0,1.2) node[above] {$\widehat{z}$};
   \end{scope}

	 
 \end{tikzpicture}
 }
 \end{center}


% TODO T-like gates, rotations on Y or X axis

\paragraph{Inverse T gate}Hermitian conjugate of the T gate.
\index{inverse T gate}
\[
T^\dagger & = Z^{-\frac{1}{4}} 
\\ 
\notag & = 
\begin{bmatrix} 
1 & 0 \\ 0 & e^{-i\tfrac{\pi}{4}} 
\end{bmatrix}
\\ \notag
& \simeq R_z(\tfrac{pi}{4})
\]
\begin{center}
\adjustbox{scale=0.8}{\begin{quantikz}[thin lines, column sep=0.75em,row sep={2.5em,between origins}]
& \gate{T^\dagger} & \qw
\end{quantikz}
}
\end{center}


An eighth turn clockwise about the \axis{z} axis.
\begin{center}
\adjustbox{scale=0.75}{
 \begin{tikzpicture}[scale=1.5]
   \begin{scope}[canvas is zy plane at x=0]
     \draw (0,0) circle (1cm);
     %\draw[ultra thin] (-1,0) -- (1,0) (0,-1) -- (0,1);
     \draw[->] (0,0) -- (1.35,0) node[below ] {$\widehat{x}$};
   \end{scope}

   \begin{scope}[canvas is zx plane at y=0]
     \draw (0,0) circle (1cm);
     %\draw (-1,0) -- (1,0) (0,-1) -- (0,1);
     \draw[->] (0,0) -- (0,1.2) node[right] {$\widehat{y}$};
   \end{scope}

   \begin{scope}[canvas is xy plane at z=0]
     \draw (0,0) circle (1cm);
	%\draw (-1,0) -- (1,0) (0,-1) -- (0,1);
	\draw[->] (0,0) -- (0,1.2) node[above] {$\widehat{z}$};
   \end{scope}

 
    
   \draw[->] (1.5,0) -- node[above] {T} ++(1.0, 0) ;
 \end{tikzpicture}
 \begin{tikzpicture}[scale=1.5]
   \begin{scope}[canvas is zy plane at x=0]
     \draw (0,0) circle (1cm);

   \end{scope}

   \begin{scope}[canvas is zx plane at y=0]
     \draw (0,0) circle (1cm);
     %\draw (-1,0) -- (1,0) (0,-1) -- (0,1);
     \draw[->] (0,0) -- (0.907,-0.907) node[left] {$\widehat{x}$};
     \draw[->] (0,0) -- (0.907,0.907) node[right] {$\widehat{y}$};     
   \end{scope}

   \begin{scope}[canvas is xy plane at z=0]
     \draw (0,0) circle (1cm);
	%\draw (-1,0) -- (1,0) (0,-1) -- (0,1);
	\draw[->] (0,0) -- (0,1.2) node[above] {$\widehat{z}$};
   \end{scope}	 
 \end{tikzpicture}
 }
 \end{center}





\subsection{Global phase}

\paragraph{Global phase gate} (phase-shift)~\cite{Barenco1995b,???,???}
\index{phase}
\index{global phase gate}
\index{phase shift gate}
\[
\label{Ph}
\Gate{Ph}(\alpha) &= 􏰔 e^{i\alpha} I  \\
& = \begin{bmatrix} e^{i\alpha} & 0 \\ 0 & e^{i\alpha} \end{bmatrix}
\notag
\]
\begin{center}
\adjustbox{scale=0.8}{\begin{quantikz}[thin lines, column sep=0.75em,row sep={2.5em,between origins}]
& \gate{\text{Ph}(\alpha)} & \qw
\end{quantikz}
}
\end{center}
To shift the global phase we multiply the quantum state by a scalar, so it is not necessary to assign a phase shift to any particular qubit. But on those occasions where we want to keep explicit track of the phase in a circuit, it is useful to assign a global phase shift to a particular qubit and temporal location, e.g.\
\begin{center}
\adjustbox{scale=0.8}{\begin{quantikz}[thin lines, column sep=0.75em,row sep={2.5em,between origins}]
& \gate{R_x(\theta)} & \qw
\end{quantikz}
}
$=$
\adjustbox{scale=0.8}{\begin{quantikz}[thin lines, column sep=0.75em,row sep={2.5em,between origins}]
& \gate{\text{Ph}(-\frac{\theta}{2})} & \gate{X^{\frac{\theta}{\pi}}} & \qw
\end{quantikz}
}
\end{center}

This gate was originally called the phase-shift gate~\cite{Barenco1995b}, but unfortunately the 1-qubit gate that shifts the phase of the 1 state relative the the zero state is also called the phase-shift gate~\eqref{phaseshift},
which is potentially confusing. 


\paragraph{Omega gate}~\cite{???,???}
\index{omega gate}
\[
\label{omega}
\omega^k &= \text{Ph}(\tfrac{\pi}{4}k) 
\\
& = \begin{bmatrix} e^{i\frac{\pi}{4}k} & 0 \\ 0 & e^{i\frac{\pi}{4}k} \end{bmatrix}
\notag
\]
An alternative parameterization of a global phase shift. Note that this gate is an eight root of the identity,  $\omega^8=I$. This gate, with integer powers, crops up when constructing the 1-qubit Clifford gates from Hadamard and S gates, since $SHSHSH=\omega$ (see p.~\pageref{cliffordomega}).  
\index{global phase gate}


% [TODO: Remove figures to separate files]
% [TODO: Am I using the terms 'phase' and 'phase factor' consistantly]


% TODO: Talk about this when do 1-qubit deke
%    \paragraph{QASM U3 gate}
%    ...

